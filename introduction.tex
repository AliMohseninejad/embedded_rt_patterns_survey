\setlength\parindent{24pt}
\section{مقدمه}

\begin{RTL}
این گزارش به طور مفصل به توضیح الگوهای معرفی‌شده در مقالات و کتب مختلف
در حوزه سیستم‌های نهفته و بی‌درنگ می‌پردازد.
برای درک عمیق‌تر این الگوها، باید ابتدا مشخص شود که منظور
از سیستم‌های نهفته بی‌درنگ چیست.
سیستم‌های نهفته در بخش‌های زیادی از زندگی روزمره وجود دارند؛
به طور مثال سیستم‌های رادیویی، سیستم‌های ناوبری، سیستم‌های تصویربرداری.
به طور کلی یک سیستم نهفته را می‌توان اینگونه تعریف کرد،:
«یک سیستم کامپیوتری که به طور مشخص برای انجام یک کار در دنیای واقعی
تخصیص داده‌شده و هدف آن ایجاد یک محیط کامپیوتری
با کاربری عام نیست» \cite{ref1}.
یک دسته مهم از سیستم‌های نهفته، سیستم‌های بی‌درنگ هستند.
«سیستم‌های بی‌درنگ، سیستم‌هایی
هستند که در آن‌ها قیدهای زمانی مشخص باید برآورده شوند
تا سیستم بتواند به درستی کار کند» \cite{ref1}.
\end{RTL}

\begin{RTL}
حال که مفهوم سیستم‌های نهفته بی‌درنگ را دریافتیم، باید تعریفی از الگو در این سیستم‌ها
ارائه دهیم. منابع متنوع تعاریف متفاوتی از الگوها ارائه کرده‌اند و بسیاری از آن‌ها
این تعریف را به الگوهای طراحی محدود می‌کنند \cite{ref1}.
هدف این گزارش تقسیم‌بندی الگوهای نرم‌افزاری به طور کلی نیست و صرفا می‌خواهیم
الگوهای مورد استفاده در سیستم‌های نهفته و بی‌درنگ را بررسی کنیم.
\lr{Zalewski} \cite{ref2} می‌گوید:
«یک الگو یک مدل یا یک قالب نرم‌افزاری است که به فرایند ایجاد نرم‌افزار کمک می‌کند.»
این تعریف در عین سادگی، جامع است؛ به طوری که الگوهای طراحی، معماری و فرایندی
را در خود شامل می‌شود. با این حال این مقاله نیز مانند بسیاری از دیگر مقالات،
تعریف جدیدی از الگوها در سیستم‌های نهفته بی‌درنگ ارائه نکرده‌اند و برای تعریف آن به
تعریف \lr{Gamma} و دیگران \cite{ref3} از الگوهای طراحی ارجاع داده‌اند.
\end{RTL}

\begin{RTL}
گزارش پیش رو ابتدا در فصل \ref{survey} به مطالعه کارهای
پیشین در حوزه الگوهای سیستم‌های نهفته بی‌درنگ می‌پردازد.
ساختار ارائه‌شده در این بخش به صورت خطی، کتاب‌ها و مقالات
بیان شده را بررسی کرده و الگوهای بیان‌شده از طرف ایشان را
با همان ساختار و دسته‌بندی مورد نظر آن منبع ذکر کرده‌است.
\end{RTL}


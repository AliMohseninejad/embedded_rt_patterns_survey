\subsection{الگوهای معماری}
\begin{RTL}
با توجه به الگوهایی که به عنوان الگوی معماری در ادامه می‌آوریم،
الگوها را به ۶ دسته تقسیم می‌کنیم.
\end{RTL}
\begin{enumerate}
    \item \lr{From Mud to Structure}
    \item \lr{Distributed Systems}
    \item \lr{Interactive Systems}
    \item \lr{Adaptable Systems}
    \item \lr{Safe Channel-based Systems}
    \item \lr{Safe Systems without Channel}
\end{enumerate}
\begin{RTL}
همانطور که در بالا دیده می‌شود، چهار دسته اول تقسیم‌بندی را براساس
\cite{ref6} آورده و دو دسته بعدی را خودمان معرفی کردیم.
\end{RTL}
\subsubsection{\lr{From Mud to Structure}}
\begin{itemize}
\item \nameref{archLayerSec}: % from mud to structure
با توجه به تعریف ارائه‌شده، این الگو با بیان ساختار یک سیستم و نحوه
ارتباط زیربخش‌های آن با هم، در دسته الگوهای معماری قرار می‌گیرد.
لازم به ذکر است که این الگو همان الگوی \lr{Layers} است
که \lr{Buschmann} و غیره \cite{ref6} گفته‌اند.
\item \nameref{arch5LayerSec}: % from mud to structure
این الگو نیز یکی از حالت‌های خاص \nameref{archLayerSec}
است و در همین دسته قرار می‌گیرد.
\item \nameref{archChannelSec}: % from mud to structure
این الگو در \cite{ref4} به عنوان یک الگوی معماری و
در \cite{ref1} به عنوان یک الگوی طراحی معرفی شده‌است.
طبق تعریف، این الگو با استفاده از یک زنجیره از اشیا می‌تواند توابعی را روی
داده، از ابتدا (\lr{Source}) تا انتها (\lr{Sink}) انجام دهد.
این رفتار بسیار شبیه الگوی \lr{Pipes and Filters} در \cite{ref6}
است. در این الگو اگر کانال یک عنصر درشت‌دانه باشد و جزوی از یک زیربخش
قابل تکرار در سیستم باشد، می‌توان این الگو را در دسته الگوهای معماری قرار داد.
\item \nameref{archRecContainSec}: % from mud to structure
این الگو با تعریف یک زیربخش سیستمی به عنوان عنصر اصلی، مي‌گوید که
زیربخش‌ها همگی با صورت تودرتو می‌توانند درون یکدیگر قرار بگیرند و سیستم را
تشکیل دهند. این الگو به دلیل درشت‌دانه بودن ساختار اجزای تشکیل‌دهنده،
در این دسته قرار می‌گیرد.
\item \nameref{archHierContSec}: % from mud to structure
این الگو نیز شباهت زیادی با \nameref{archRecContainSec} داشته
و در این دسته قرار می‌گیرد.
\end{itemize}

\subsubsection{\lr{Distributed Systems}}
\begin{itemize}
\item \nameref{distrSharedMemSec}: % distributed systems
این الگو با به اشتراک‌گذاری یک واحد حافظه بین چندین پردازنده،
امکان استفاده پردازنده‌ها از یک حافظه مشترک را فراهم می‌سازد.
اجزای تشکیل‌دهنده این الگو، پردازنده‌ها و حافظه‌ها هستند که اجزای
درشت‌دانه سیستم‌های نهفته به ‌شمار می‌روند. از این رو این الگو در
دسته الگوهای معماری قرار می‌گیرد.
\item \nameref{distrRemMethodCallSec} % distributed
این الگو با فراهم‌کردن یک پروتکل ارتباطی بین کلاینت و سرور،
در یک ساختار توزیع‌شده اجازه فراخوانی توابع سرور توسط کلاینت را
فراهم می‌سازد. از این رو این الگو در دسته الگوهای معماری قرار می‌گیرد.
% \item \nameref{distrObserverSec}:
\item \nameref{distrDataBusSec}: % distributed
این الگو با ایجاد یک گذرگاه داده، اجازه می‌دهد زیربخش‌های یک
سیستم با استفاده از آن، با یکدیگر ارتباط داشته باشند
و بتوانند روی این گذرگاه، داده‌های مورد نیاز را ارسال
و دریافت کنند. این الگو در سیستم‌های نهفته کاربردهای زیادی دارد
و معمولا بین سخت‌افزارهای مجزا استفاده می‌شود.
این الگو نیز جزو دسته الگوهای معماری است.
% \item \nameref{distrProxySec}
\item \nameref{distrBrokerSec}: % distributed
این الگو کاملا مشابه الگوی \lr{Broker} در \cite{ref6} است
و در دسته الگوهای معماری قرار می‌گیرد.
\end{itemize}

\subsubsection{\lr{Interactive Systems}}
\begin{itemize}
\item \nameref{ZalewskiFramework}:
این الگو با ایجاد یک ساختار معماری با در نظر داشتن رابط کاربری و
ارتباط بخش‌های مختلف سیستم و سیستم‌های مختلف با هم، یک معماری
قابل استفاده برای کاربران ایجاد می‌کند که استفاده از سیستم نهفته را ممکن
می‌سازد. 
\end{itemize}

\subsubsection{\lr{Adaptable Systems}}
\begin{itemize}
\item \nameref{archMicrokernelSec}: % adaptable systems
این الگو نیز همان الگوی \lr{Microkernel} است
که در \cite{ref6} گفته‌شده.
\item \nameref{archVirtMachineSec}: % adaptable systems
این الگو یک سیستم کامل را بر روی یک ماشین مجازی پیاده‌سازی‌کرده و
با هدف تطبیق‌پذیری بالا طراحی شده‌است. این الگو نحوه استقرار یک سیستم
نرم‌افزاری بر روی پلتفرم‌های مختلف را بررسی می‌کند و مشخصا یک الگوی معماری است.
\item \nameref{archCompBasedSec}: % adaptable systems
این الگو به ایجاد یک سیستم نرم‌افزاری مبتنی بر اجزای قابل استفاده مجدد تکیه دارد
و یک الگوی معماری است.
% \item \nameref{archROOMSec}:
\end{itemize}

\subsubsection{\lr{Safe Channel-based Systems}}
\begin{itemize}
\item \nameref{archSafeHomoRedundancySec}: % SafetyCh
این الگو بر پایه الگوی \nameref{archChannelSec} ساخته‌شده
و صرفا برای منظور امنیت، یک کانال اضافی نیز تعریف می‌کند.
\item \nameref{archSafeTripModRedunSec}: % SafetyCh
مشابها در این الگو نیز از کانال به تعداد سه مرتبه استفاده شده‌است.
\item \nameref{archSafeHeteroRedundancySec}: % SafetyCh
این الگو بر پایه الگوی \nameref{archChannelSec} ساخته‌شده
و صرفا برای منظور امنیت، یک کانال اضافی از نوعی دیگر نیز تعریف می‌کند.
\item \nameref{archSafeMonActSec}: % SafetyCh
این الگو نیز با استفاده از کانال‌ها یک ساختار معماری امن ارائه می‌کند.
\item \nameref{archSafeSanityChkSec}: % SafetyCh
این الگو نیز با استفاده از کانال‌ها باعث ایجاد یک ساختار امن می‌شود.
\item \nameref{archSafeWatchDogSec}: % SafetyCh
این الگو نیز با استفاده از کانال‌ها باعث ایجاد یک ساختار امن می‌شود.
\item \nameref{archSafeSafetyExecSec}: % SafetyCh
این الگو نیز با استفاده از کانال‌ها باعث ایجاد یک ساختار امن می‌شود.
\item \nameref{ArmoushHWMOutNSec}: % SafetyCh
این الگو نسخه تعمیم‌یافته
\nameref{archSafeTripModRedunSec} است.
\item \nameref{ArmoushMix3LvlSafeMonSec} % SafetyCh
این الگو با استفاده از تکرار نرم‌افزار به علاوه استفاده از کانال‌ها، امنیت را
فراهم می‌کند.
% \item \nameref{archSafeProtectSingleChSec}: % SafetyCh
% این الگو بر روی \nameref{archChannelSec} ساخته‌شده
% و صرفا تعدادی چک بر روی داده‌ها اضافه‌کرده است.
\end{itemize}

\subsubsection{\lr{Safe Systems without Channel}}
\begin{itemize}
\item \nameref{ArmoushSWNVerProgSec} % Safety no Ch
این الگو با ایجاد تکرار در نرم‌افزار، به امنیت و قابلیت اطمینان کمک می‌کند.
با توجه به این که در این الگو، اجزا در سطح زیرسیستم هستند،
می‌توان این الگو را در دسته الگوهای معماری درنظر گرفت.
\item \nameref{ArmoushSWRecoverBlockSec} % Safety no Ch
مشابه الگوی قبلی با تکرار نرم‌افزار امنیت را فراهم می‌کند.
\item \nameref{ArmoushSWAccVoteSec} % Safety no Ch
مشابه الگوی قبلی با تکرار نرم‌افزار امنیت را فراهم می‌کند.
\item \nameref{ArmoushSWNSelfChkProgSec} % Safety no Ch
مشابه الگوی قبلی با تکرار نرم‌افزار امنیت را فراهم می‌کند.
\item \nameref{ArmoushSWRecoverBlockBackVotSec} % Safety no Ch
مشابه الگوی قبلی با تکرار نرم‌افزار امنیت را فراهم می‌کند.
\end{itemize}
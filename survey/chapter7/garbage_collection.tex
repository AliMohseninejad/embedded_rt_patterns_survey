\subsubsection{الگوی \lr{Garbage Collection}}
\label{memGarbCollectionSec}
\begin{RTL}
الگوی \lr{Garbage Collection} \cite{ref4} به مشکلات
مدیریت حافظه مانند نشت حافظه و \lr{Dangling Pointer} با
خودکارسازی فرآیند آزادسازی حافظه پرداخته و نیاز به آزادسازی صریح
حافظه توسط برنامه‌نویسان را حذف می‌کند. این الگو به طور
قابل‌توجهی مشکلات مرتبط با حافظه را کاهش داده و برای سیستم‌های
با دسترسی بالا که نیاز به اجرای طولانی‌مدت بدون راه‌اندازی مجدد دارند،
ایده‌آل است. با این حال، این الگو به دلیل طبیعت دوره‌ای خود، سربار
زمان اجرا و عدم پیش‌بینی‌پذیری در اجرا را به همراه دارد و
مشکل تکه‌تکه شدن حافظه را حل نمی‌کند که می‌توان
آن را با استفاده از \nameref{memGarbCompatorSec} مدیریت کرد.
\end{RTL}
\begin{figure}[h!]
\centering
\begin{tikzpicture}
    \lr{
            \umlclass[x=-1, y=-12]{Collectable}{
            \lr{isLive}
            }{
                }   
                \umlclass[x=-3, y=-15]{Client}{     
                }{
                    } 
        \begin{umlpackage}[x=5, y=-10]{MemoryManagmentSystem}
                \umlclass[y=-1]{MemoryBlock}{}{
                        }   
                \umlclass[y=-5]{FreeBlockList}{}{
                            } 
                \umlclass[y=-8]{GarbegeCollector}{
                }{
                                } 
                \end{umlpackage}
                \umlinherit[]{Client}{Collectable}
                \umlassoc[mult1=*, attr2=*|usedBlock]{Client}{MemoryBlock}
                \umluniassoc[mult1=1, mult2=1]{GarbegeCollector}{FreeBlockList}
                \umluniassoc[mult1=1, attr2=*|rootObject, geometry=-|, pos2=1.8]{GarbegeCollector}{Client}
                \umluniaggreg[mult1=1, attr2=*|freeBlock]{FreeBlockList}{MemoryBlock}
                \umluniassoc[mult1=*, attr2=*|derivedObject, angle1=130, angle2=170]{Client}{Client}
}
\end{tikzpicture}
\caption{دیاگرام کلاس \lr{Garbage Collection}}
\label{memGarbCollectionClassDiag}
\end{figure}
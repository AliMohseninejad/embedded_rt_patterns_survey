\subsubsection{الگوی \lr{Fixed Sized Buffer}}
\label{memFixSizeBufSec}
\begin{RTL}
این الگو \cite{ref4}
مشکل تکه‌تکه شدن حافظه را در تخصیص حافظه پویا برطرف می‌کند،
که این مسئله برای سیستم‌های بی‌درنگ نهفته که باید برای مدت‌های طولانی
به‌طور قابل اعتماد عمل کنند، بسیار مهم است.
این الگو با استفاده از بلوک‌های حافظه با اندازه ثابت از تکه‌تکه شدن
جلوگیری می‌کند، اگرچه منجر به هدر رفتن بخشی از حافظه به دلیل
استفاده غیر بهینه می‌شود. این مصالحه معمولاً در بسیاری از
سیستم‌های عامل بی‌درنگ قابل قبول است که اغلب از
تخصیص بلوک‌های با اندازه ثابت به صورت داخلی پشتیبانی می‌کنند.
\end{RTL}
\begin{figure}[h!]
\centering
\begin{tikzpicture}
    \lr{
        \umlclass[]{Client}{
            \lr{}
        }{
            \lr{}
            }
            \umlclass[y=-3]{ObjectFactory}{
        }{
            }
            \umlclass[y=-6]{HeapManager}{}{
                }
                \umlclass[x=5, y=-6]{SizedHeap}{
                }{
        } 
        \begin{umlpackage}[x=12, y=-6]{MemorySegment}
        \umlclass[y=-1]{FreeBlockList}{}{
            }   
    \end{umlpackage}   
    \umluniassoc[mult1=*, mult2=*]{Client}{ObjectFactory}
    \umluniassoc[mult1=*, mult2=1]{ObjectFactory}{HeapManager}
    \umluniassoc[mult1=1, mult2=*]{HeapManager}{SizedHeap}
    \umluniassoc[mult1=1, attr2=*|freeList, anchor1=-20]{SizedHeap}{FreeBlockList}
    \umlunicompo[mult1=1, mult2=1, anchor1=10, anchor2=155]{SizedHeap}{MemorySegment}
}
\end{tikzpicture}
\caption{دیاگرام کلاس \lr{Fixed Sized Buffer}}
\label{memFixSizeBufClassDiag}
\end{figure}
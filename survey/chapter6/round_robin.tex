\subsubsection{الگوی \lr{Round Robin}}
\label{archConRoundRobSec}
\begin{RTL}
الگوی \lr{Round Robin} \cite{ref4} با
دادن فرصت به همه وظایف برای پیشرفت،
اطمینان از عدالت در اجرای وظایف را فراهم می‌کند و برای سیستم‌هایی مناسب
است که پیشرفت کلی سیستم مهم‌تر از برآورده شدن ددلاین‌های خاص است.
برخلاف \nameref{archConCyclicExecSec}، الگوی
\lr{Round Robin} از پیش‌دستی زمانی
استفاده می‌کند و مانع از متوقف شدن سیستم توسط یک وظیفه
نادرست می‌شود. با این حال، این الگو محدودیت‌هایی مانند پاسخ‌دهی
نامطلوب به رویدادها و عدم پیش‌بینی‌پذیری در شرایط بار بسیار زیاد
را دارد. با این که این الگو در مقایسه با \nameref{archConCyclicExecSec}
بهتر به تعداد بیشتری از
وظایف مقیاس‌پذیر است، اما با افزایش تعداد وظایف، می‌تواند منجر به تشدید وظایف شود
و زمان مؤثر هر وظیفه کاهش یابد. مکانیزم‌های اشتراک‌گذاری داده‌ها ابتدایی هستند
و پیاده‌سازی مدل‌های پیچیده را دشوار می‌کنند.
\end{RTL}
\begin{figure}[h!]
\centering
\begin{tikzpicture}
    \lr{
        \umlclass[y=4]{Timer}{
        }{
            }
            \umlclass[]{Scheluer}{
            \lr{}
        }{
            \lr{switchTask()}
            }
            \umlclass[x=4.5, y=4]{TaskControlBlock}{
                \lr{startAddr}\\
                \lr{RentryPoint}
            }{          
    }
            \umlclass[x=10, y=4]{Stack}{
                \lr{baseAddr}\\
                \lr{Top}
            }{  
    }       
    \umlclass[x=9]{AbstractThread}{}{  
        \lr{run()}      
}       
\umlclass[x=8, y=-3]{ConcreteThread}{}{      
}       
    \umlassoc[attr1=1|timeScliceSource, mult2=1]{Timer}{Scheluer}
    \umlunicompo[mult1=1, attr2=TCB|*, geometry=-|, anchor1=30, pos2=1.7]{Scheluer}{TaskControlBlock}
    \umlunicompo[mult1=1, mult2=1, anchor1=47]{AbstractThread}{Stack}
    \umlassoc[mult1=1, attr2=ordered|*]{Scheluer}{AbstractThread}
    \umlassoc[mult1=1, mult2=1]{TaskControlBlock}{Stack}
    \umlinherit[]{ConcreteThread}{AbstractThread}
    }  
\end{tikzpicture}
\caption{دیاگرام کلاس \lr{Round Robin}}
\label{archConRoundRobClassDiag}
\end{figure}
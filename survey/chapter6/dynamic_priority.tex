\subsubsection{الگوی \lr{Dynamic Priority}}
\label{archConDynPriorSec}
\begin{RTL}
الگوی \lr{Dynamic Priority}، اولویت وظایف را بر اساس فوریت
در زمان اجرا تنظیم می‌کند و معمولاً از استراتژی نزدیک‌ترین ضرب‌الاجل
استفاده می‌کند، جایی که وظیفه‌ای که نزدیک‌ترین ضرب‌الاجل را دارد
بالاترین اولویت را دریافت می‌کند. این روش بهینه است زیرا اگر
وظایف بتوانند توسط هر الگوریتمی زمان‌بندی شوند، می‌توانند توسط این
الگوریتم نیز زمان‌بندی شوند. با این حال، این الگو ناپایدار است،
به این معنی که پیش‌بینی اینکه کدام وظایف در شرایط بار زیاد شکست می‌خورند،
ممکن نیست. این الگو برای سیستم‌های پیچیده با وظایف تقریباً برابر از
نظر اهمیت و جایی که تحلیل استاتیک غیرممکن است، مناسب است.
در مقابل، \nameref{archConStaticPriorSec} برای سیستم‌های
ساده‌تر که وظایف و زمان‌بندی آن‌ها می‌تواند به دقت شناخته و برنامه‌ریزی شود،
بهتر است.
\end{RTL}
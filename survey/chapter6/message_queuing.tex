\subsubsection{الگوی \lr{Message Queuing}}
\label{archConMessageQueSec}
\begin{RTL}
الگوی صف پیام \cite{ref4} روشی ساده برای ارتباط بین \lr{Thread} ها فراهم می‌کند.
علی‌رغم اینکه این روش نسبتاً سنگین برای اشتراک‌گذاری اطلاعات است، اما به‌طور گسترده
استفاده می‌شود زیرا توسط بیشتر سیستم‌عامل‌ها پشتیبانی می‌شود و به راحتی
قابل اثبات صحت است. این الگو از مشکلات \lr{Mutal Exclusion}
جلوگیری می‌کند زیرا هیچ منبع اشتراکی نیاز به محافظت ندارد، که همگام‌سازی را ساده کرده
و یکپارچگی داده‌ها را تضمین می‌کند. در این الگو، اطلاعات به جای ارجاع،
به صورت مقدار پاس داده می‌شوند و از مسائل فساد داده‌ای
که در سیستم‌های همزمان رایج است، جلوگیری می‌شود. با این حال،
این روش برای پردازش ساختارهای داده بزرگ کارایی کمتری دارد
و اشتراک‌گذاری اطلاعات بسیار کارآمد را تسهیل نمی‌کند.
(این الگو همان \nameref{scheduleQueuingSec} است که در
\cite{ref1} گفته شده.)
\end{RTL}
\subsubsection{الگوی \lr{Cyclic Execution}}
\label{archConCyclicExecSec}
\begin{RTL}
الگوی \lr{Cyclic Execution} در سیستم‌های کوچک
یا سیستم‌هایی که نیاز به اجرای قابل پیش‌بینی دارند،
مانند سیستم‌های هوانوردی، به طور گسترده‌ای استفاده می‌شود.
این الگو با سادگی و پیش‌بینی‌پذیری خود، پیاده‌سازی آسانی دارد و برای محیط‌های محدود به
حافظه که یک سیستم‌عامل بی‌درنگ کامل عملی نیست، مناسب است. این الگو وظایف
را در یک حلقه ثابت و تکراری اجرا می‌کند و اطمینان می‌دهد که هر
وظیفه به نوبت اجرا می‌شود. سادگی آن نقطه قوت اصلی آن
است، اما انعطاف‌پذیری ندارد و برای رسیدگی به رویدادهای با ضرب‌الاجل‌های
محدود بهینه نیست. وظایف نمی‌توانند در زمان اجرا اضافه یا
حذف شوند و سیستم به تنظیمات زمانی حساس است.
وظایف نادرست می‌توانند کل سیستم را مختل کنند و
الگو در شرایط بار زیاد ناپایدار است. با وجود محدودیت‌ها، برای
سیستم‌های کوچک و پایدار با دینامیک‌های قابل درک مناسب است.
(این الگو همان \nameref{scheduleCyclicExecSec} است که
در \cite{ref1} گفته‌شده.)
\end{RTL}
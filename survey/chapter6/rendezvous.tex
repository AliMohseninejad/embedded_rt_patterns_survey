\subsubsection{الگوی \lr{Rendezvous}}
\label{archConRendezSec}
\begin{RTL}
الگوی \lr{Rendezvous} \cite{ref4} نسخه
ساده‌تری از \nameref{archConGuardCallSec} است
که برای همگام‌سازی \lr{Thread}ها یا اجازه اشتراک‌گذاری داده‌ها بین آنها استفاده می‌شود.
این الگو از یک شیء \lr{Rendezvous} برای مدیریت همگام‌سازی استفاده می‌کند
که ممکن است شامل داده‌های اشتراکی یا فقط اعمال سیاست‌های همگام‌سازی باشد.
ساده‌ترین شکل آن، الگوی \lr{Thread Barrier} است که
\lr{Thread}ها را بر اساس تعداد مشخصی که در یک نقطه ثبت‌نام می‌کنند،
همگام می‌کند. پیش‌شرط‌ها برای همگام‌سازی باید برآورده شوند که
اغلب توسط ماشین‌های حالت در زبان‌های طراحی مانند \lr{UML} مدیریت می‌شوند.
این الگو تضمین می‌کند که \lr{Thread}ها منتظر می‌مانند
تا همه شرایط برآورده شود و سپس ادامه می‌دهند.
این الگو بسیار انعطاف‌پذیر است، برای نیازهای همگام‌سازی پیچیده کاربرد دارد
و به خوبی با تعداد زیادی از \lr{Thread}ها و شروط، مقیاس‌پذیر است.
(این الگو همان \nameref{scheduleRendezvousSec} است
که در \cite{ref1} گفته‌شده.)
\end{RTL}
\begin{figure}[h!]
\centering
\begin{tikzpicture}
    \lr{
        \umlclass[y=4]{SynchPolicy}{
        }{
            }
            \umlclass[]{Rendezvous}{
            \lr{}
        }{
            \lr{reset()}\\
            \lr{register()}\\
            \lr{release()}
            }
            \umlclass[y=-4]{Callback}{
            }{
    }
            \umlclass[x=5]{ClientThread}{  
            }{ 
            \lr{notify()}       
    }       
    \umlunicompo[mult1=1, mult2=1]{Rendezvous}{SynchPolicy}
    \umluniassoc[mult1=1, mult2=*]{Rendezvous}{Callback}
    \umluniassoc[mult1=2..*, mult2=1]{ClientThread}{Rendezvous}
    }  
\end{tikzpicture}
\caption{دیاگرام کلاس \lr{Rendezvous}}
\label{archConRendezClassDiag}
\end{figure}
\subsubsection{الگوی \lr{Guarded Call}}
\label{archConGuardCallSec}
\begin{RTL}
الگوی فراخوانی محافظت‌شده \cite{ref4} راهی برای دستیابی به همگام‌سازی به موقع بین
\lr{Thread}ها از طریق فراخوانی همزمان متدها در یک
\lr{Thread} دیگر ارائه می‌دهد و از \lr{Mutal Exclusion Semaphores} 
برای جلوگیری از فساد داده و \lr{Deadlock} استفاده می‌کند.
در حالی که ارتباطات غیرهمزمان مانند \nameref{archConMessageQueSec}
اغلب منجر به تبادل اطلاعات کندتر می‌شود، الگوی فراخوانی محافظت‌شده
با اجازه دادن به فراخوانی مستقیم متدها، زمان پاسخ‌دهی سریع‌تری را تضمین می‌کند.
این الگو زمانی که همگام‌سازی فوری مورد نیاز است بسیار مفید است،
اگرچه باید با دقت پیاده‌سازی شود تا از مشکلات \lr{Mutal Exclusion}
جلوگیری شود. اگر منابع قفل باشند، بدون تحلیل مناسب نمی‌توان پاسخ‌دهی به موقع
را تضمین کرد.
(این الگو همان \nameref{scheduleGuardedCallSec} است که در
\cite{ref1} گفته‌شده.)
\end{RTL}
\begin{figure}[h!]
\centering
\begin{tikzpicture}
    \lr{
        \begin{umlpackage}[x=-3]{ClientThread}
            \umlclass[]{Client}{}{
                    } 
        \end{umlpackage}
        \begin{umlpackage}[x=6]{ServerThread}
                \umlclass[x=1]{Server}{}{
                }   
                \umlclass[x=-3]{BoundaryObject}{}{
                } 
                \umlclass[x=1, y=-3]{SharedResource}{}{
                } 
                \umlclass[x=-3, y=-3]{Mutex}{}{} 
                \end{umlpackage}
                \umluniassoc[mult1=*, mult2=*]{Client}{BoundaryObject}
                \umlassoc[mult1=1, mult2=*]{BoundaryObject}{Server}
                \umlassoc[mult1=*, mult2=*]{Server}{SharedResource}
                \umlassoc[mult1=1, mult2=1]{Mutex}{SharedResource}
}
\end{tikzpicture}
\caption{دیاگرام کلاس \lr{Guarded Call}}
\label{archConGuardCallClassDiag}
\end{figure}
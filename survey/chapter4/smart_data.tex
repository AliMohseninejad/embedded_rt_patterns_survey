\subsubsection{الگوی \lr{Smart Data}}
\label{safeSmartDataSec}
\begin{RTL}
این الگو \cite{ref1} با تولید گاردهایی روی داده‌ها و تعریف پیش‌شرط‌هایی روی آن‌ها
در توابع مختلف تلاش می‌کند تا حد ممکن رفتار برنامه و توابع را
به یک صورت \lr{Safe} ایجاد کند. این شروط در زمان اجرا برنامه چک
می‌شوند و نه در زمان کامپایل.
\end{RTL}
\begin{figure}[h!]
\centering
\begin{tikzpicture}
    \lr{
        \umlclass[x=-1]{ErrorManager}{
            \lr{}
        }{ \lr{handleError()}
        }
            \umlclass[x=2, y=-4]{ServerClass}{
            \lr{sdt}
        }{
            \lr{setValue()}\\
            \lr{getValue()}\\
            \lr{Init()}
            }
            \umlclass[x=-1, y=-10]{ErrorCodeType}{
                \lr{NOERRORS}\\
                \lr{BELOWRANGE}\\
                \lr{ABOVERAGE}\\
                \lr{INCONSISTENTVALUE}\\
                \lr{ILLEGALUSEOFNULLPTR}\\
                \lr{INDEXOUTOFRANGE}
            }{
                }
                \umlclass[x=9, y=-2]{SmartDataType}{
                    \lr{value}\\
                    \lr{lowRange}\\
                    \lr{highRange}\\
                    \lr{errorCode}
                }{
                    \lr{Init}\\
                    \lr{checkValidity()}\\
                    \lr{setValue()}\\
                    \lr{setPrimitive()}\\
                    \lr{getPrimitive()}\\
                    \lr{getValue()}\\
                    \lr{errorHandler()}\\
                    \lr{setLowBoundary()}\\
                    \lr{setHighBoundary()}\\
                    \lr{getLowBoundary()}\\
                    \lr{getHighBoundary()}\\
                    \lr{getErrorCode()}\\
                    \lr{cmp()}\\
                    \lr{pCmp()}
    }    
    \umldep[mult1=Usage]{ErrorManager}{ErrorCodeType}
    \umluniassoc[mult2=1, anchor1=60, pos2=0.9]{SmartDataType}{ErrorManager}
    \umldep[mult2=Usage, pos2=0.6]{ServerClass}{SmartDataType}
    \umldep[mult2=Usage]{ServerClass}{ErrorCodeType}
    }
\end{tikzpicture}
\caption{دیاگرام کلاس \lr{Smart Data}}
\label{safeSmartDataClassDiag}
\end{figure}
\begin{RTL}
همانطور که در شکل \ref{safeSmartDataClassDiag}
دیده می‌شود، ساختار داده \lr{SmartDataType}، توابع زیادی برای
چک‌کردن داده خود به \lr{ServerClass} ارائه داده‌است.
این الگو با قراردادن چک‌های امنیتی بر روی ساختار داده، می‌تواند باعث امنیت
داده در زمان \lr{set}شدن شود. اما این چک‌ها خود سرباری برای انجام هر عملیات
با این داده‌ها می‌شوند.
\end{RTL}
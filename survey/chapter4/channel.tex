\subsubsection{الگوی \lr{Channel}}
\label{safeChannelSec}
\begin{RTL}
الگوی کانال \cite{ref1} از تکرار در مقیاس متوسط
تا بزرگ پشتیبانی می‌کند و به شناسایی و مدیریت خطاهای زمان اجرا کمک می‌کند.
یک کانال داده‌ها را از مرحله دریافت تا \lr{Actuation} پردازش می‌کند
و واحدی مستقل و خودکفا از عملکرد را فراهم می‌کند. ارزش این الگو
در استفاده از چندین کانال برای رفع نگرانی‌های
ایمنی و قابلیت اطمینان است. این الگو شناسایی واضح خطا
و تداوم خدمات یا دستیابی به حالت \lr{Fault-Safe} را ارائه می‌دهد.
با این حال، به حافظه، زمان پردازش و سخت‌افزار اضافی نیاز دارد.
\end{RTL}
\begin{figure}[h!]
\centering
\begin{tikzpicture}
    \lr{
        \umlclass[]{SensorDeviceDriver}{
            \lr{data}
        }{
            \lr{acquireData()}
            }
            \umlclass[x=6]{AbstractDataTransform}{
            \lr{}
        }{
            \lr{transform()}
            }
            \umlclass[x=13]{ActuatorDataDriver}{
            \lr{data}
            }{
    }
            \umlclass[x=6, y=-4]{ConcrerteDataTransform}{
            \lr{data}   
            }{        
    }       
    \umluniassoc[mult2=1]{SensorDeviceDriver}{AbstractDataTransform}
    \umluniassoc[mult2=0..1]{AbstractDataTransform}{ActuatorDataDriver}
    \umluniassoc[mult2=0..1, angle1=-0.2, angle2=30]{AbstractDataTransform}{AbstractDataTransform}
    \umlinherit[]{ConcrerteDataTransform}{AbstractDataTransform}
    }  
\end{tikzpicture}
\caption{دیاگرام کلاس \lr{Channel}}
\label{safeChannelClassDiag}
\end{figure}
\begin{RTL}
همانطور که در شکل \ref{safeChannelClassDiag} دیده می‌شود،
این الگو با ایجاد زنجیره‌ای از پردازش‌ها (\lr{DataTransform})،
داده‌ها را از سنسور دریافت کرده و در هر مرحله پردازشی روی آن انجام می‌دهد.
در نهایت داده‌های پردازش‌شده به سمت \lr{Actuator}ها هدایت می‌شود.
\end{RTL}
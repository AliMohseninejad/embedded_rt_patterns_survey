\subsubsection{الگوی \lr{Recovery Block}}
\label{ArmoushSWRecoverBlockSec}
\begin{RTL}
این الگو یک روش نرم‌افزاری تحمل خطا است که از تشخیص خطا
با آزمون‌های پذیرش و بازیابی خطا به عقب برای جلوگیری از خرابی سیستم
استفاده می‌کند. مشابه \nameref{ArmoushSWNVerProgSec}، این روش شامل ایجاد
\lr{N} نسخه نرم‌افزاری متنوع، مستقل و معادل عملکردی
از مشخصات اولیه است. این نسخه‌ها به نسخه اصلی و \lr{N-1}
نسخه ثانویه تقسیم می‌شوند. اجرای هر نسخه با یک
آزمون پذیرش دنبال می‌شود. اگر نسخه اصلی در آزمون خود شکست بخورد،
یک نسخه ثانویه \lr{Recovery Block} اجرا می‌شود
و پس از آن آزمون پذیرش انجام می‌شود. این فرآیند تا زمانی که
یک نسخه آزمون پذیرش را بگذارد یا همه نسخه‌ها شکست بخورند
و یک خرابی کلی سیستم گزارش شود، تکرار می‌شود.
این روش برای سیستم‌های با ایمنی بسیار بالا
مناسب است، زمانی که نرم‌افزار بسیار قابل اعتماد
و ایمنی نیاز است، امکان ساخت آزمون پذیرش برای اطمینان از
عملکرد صحیح نرم‌افزار و تشخیص خروجی‌های ممکن نادرست
وجود دارد، تیم‌های مستقل برای توسعه نسخه‌های مختلف موجود
هستند و هزینه بالای توسعه نسخه‌های متعدد قابل
تحمل است. معایب اصلی شامل وابستگی
زیاد به کیفیت آزمون پذیرش، احتمال قطع سرویس در
حین بازیابی و مشکلات مشترک با \nameref{ArmoushSWNVerProgSec} مانند
هزینه بالای توسعه، پیچیدگی توسعه نسخه‌های مستقل و وابستگی به مشخصات
اولیه که ممکن است خطاهای نرم‌افزاری را به همه نسخه‌ها منتقل کند، می‌باشد.
\end{RTL}
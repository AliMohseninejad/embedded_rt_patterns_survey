\subsubsection{الگوی \lr{N-Self Checking Programming}}
\label{ArmoushSWNSelfChkProgSec}
\begin{RTL}
این الگو \cite{ref5}
یک روش نرم‌افزاری تحمل خطای بسیار پرهزینه است
که بر تنوع طراحی نرم‌افزار و خود بررسی از طریق تکرار تأکید دارد.
این روش شامل تولید مستقل حداقل چهار ماژول نرم‌افزاری معادل
عملکردی از مشخصات اولیه است.
این نسخه‌ها به گروه‌هایی به نام اجزا مرتب می‌شوند،
که هر جزء شامل دو نسخه و یک الگوریتم مقایسه برای بررسی صحت نتایج است.
در طول اجرا، یک جزء به طور فعال خدمت مورد نیاز را ارائه
می‌دهد، در حالی که اجزای دیگر به عنوان یدک‌های آماده عمل می‌کنند.
برای اطمینان از تحمل خطا برای یک خطا، حداقل چهار
نسخه باید بر روی چهار واحد سخت‌افزاری اجرا شوند، که آن
را به پرهزینه‌ترین روش در مقایسه با سایر روش‌ها تبدیل
می‌کند. این الگو برای توسعه نرم‌افزار تحمل خطا برای
سیستم‌های بسیار بحرانی از نظر ایمنی مناسب است، جایی که نیاز
به نرم‌افزار بسیار قابل اعتماد است، هزینه بالای پیاده‌سازی‌های
متعدد قابل تحمل است، تیم‌های مستقل برای
توسعه نسخه‌های مختلف موجود هستند و امکان
استفاده از واحدهای سخت‌افزاری اضافی برای اجرای
این نسخه‌ها به صورت موازی وجود دارد. معایب اصلی این الگو
شامل وابستگی زیاد به مشخصات اولیه که ممکن است خطاها را به همه
نسخه‌ها منتقل کند، تعداد بالای نسخه‌های متنوع و
ماژول‌های سخت‌افزاری مورد استفاده در مقایسه با سایر
الگوها که همان تعداد خطا را تحمل می‌کنند،
و پیچیدگی توسعه \lr{N} نسخه‌ مستقل و معادل عملکردی است.
\end{RTL}
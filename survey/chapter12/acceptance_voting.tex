\subsubsection{الگوی \lr{Acceptance Voting}}
\label{ArmoushSWAccVoteSec}
\begin{RTL}
الگوی رأی‌گیری پذیرش \lr{(AVP)} یک روش نرم‌افزاری تحمل خطای
ترکیبی است که عناصر \nameref{ArmoushSWNVerProgSec} \lr{(NVP)}
و \nameref{ArmoushSWRecoverBlockSec} \lr{(RB)} را
ترکیب می‌کند. این روش شامل تولید مستقل \lr{N (N ≥ 2)}
ماژول‌ نرم‌افزاری معادل از نظر عملکردی از مشخصات اولیه است.
این نسخه‌ها به صورت موازی اجرا می‌شوند و همان وظیفه را بر روی
همان ورودی انجام می‌دهند تا \lr{N} خروجی تولید کنند.
هر خروجی تحت آزمون پذیرش قرار می‌گیرد تا از صحت آن اطمینان
حاصل شود. خروجی‌هایی که آزمون پذیرش را می‌گذرانند، به عنوان ورودی
به یک رأی‌گیر پویا استفاده می‌شوند که خروجی صحیح
را بر اساس یک طرح رأی‌گیری تعیین می‌کند.
این الگو برای توسعه نرم‌افزار تحمل خطا برای سیستم‌های بسیار بحرانی
از نظر ایمنی مناسب است، جایی که نیاز به
نرم‌افزار بسیار قابل اعتماد است، آزمون‌های پذیرش
قابل ساخت هستند، هزینه بالای پیاده‌سازی‌های متعدد قابل تحمل
است، تیم‌های مستقل برای توسعه نسخه‌های مختلف موجود هستند و
امکان استفاده از واحدهای سخت‌افزاری اضافی برای اجرای نسخه‌ها
به صورت موازی وجود دارد. مشابه روش
اصلی \nameref{ArmoushSWNVerProgSec}، معایب اصلی \lr{AVP} شامل
تلاش برای توسعه نسخه‌های نرم‌افزاری متنوع و وابستگی زیاد به مشخصات
اولیه است که ممکن است خطاها را به همه نسخه‌ها منتقل کند.
با این حال، آزمون پذیرش یک اقدام اضافی برای تشخیص خطاهای
وابسته فراهم می‌کند و مشکل خطاهای وابسته در این الگو کمتر بحرانی است.
\end{RTL}
\subsubsection{الگوی \lr{N-Version Programming}}
\label{ArmoushSWNVerProgSec}
\begin{RTL}
الگوی \lr{N-Version Programming (NVP)} یک روش نرم‌افزاری تحمل خطا
است که بر تنوع نرم‌افزار و مخفی‌سازی خطاها متکی است.
این روش شامل ایجاد \lr{N} نسخه نرم‌افزاری معادل عملکردی
\lr{(N ≥ 2)} به‌صورت مستقل از مشخصات اولیه است.
این نسخه‌ها به‌طور موازی اجرا می‌شوند و وظیفه یکسانی را با
ورودی یکسان انجام می‌دهند تا \lr{N} خروجی تولید کنند.
در این الگو، یک رأی‌گیر برای تعیین خروجی صحیح با استفاده از نتایج
\lr{N} نسخه به کار می‌رود. \lr{NVP} برای سیستم‌های با ایمنی بسیار بالا مناسب است،
زمانی که نیاز به نرم‌افزار بسیار قابل اعتماد وجود دارد،
هزینه بالای توسعه نسخه‌های متعدد قابل تحمل است، تیم‌های مستقل برای توسعه نسخه‌های
مختلف موجود هستند و واحدهای سخت‌افزاری تکراری برای اجرای این
نسخه‌ها به‌صورت موازی قابل استفاده هستند. با این حال، \lr{NVP}
دارای معایبی مانند پیچیدگی و هزینه بالای توسعه نسخه‌های مستقل
و وابستگی به مشخصات اولیه است که می‌تواند خطاها را به تمامی نسخه‌ها
منتقل کند و ایمنی و قابلیت اطمینان سیستم را تحت تأثیر قرار دهد.
\end{RTL}
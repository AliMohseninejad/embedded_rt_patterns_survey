\subsubsection{الگوی \lr{Broker}}
\label{distrBrokerSec}
\begin{RTL}
الگوی بروکر \cite{ref4}
یک نسخه متقارن از \nameref{distrProxySec} است که برای شرایطی
طراحی شده که مکان کلاینت و سرورها در زمان طراحی مشخص نیست. این الگو
یک بروکر را معرفی می‌کند، که یک مخزن ارجاع شیء است
و برای هر دو کلاینت و سرورها قابل مشاهده است و
به کلاینت‌ها در یافتن سرورها کمک می‌کند. این کار، امکان استقرار
معماری‌های متقارن مانند تعادل بار پویا را فراهم می‌کند. الگوی بروکر
مسائل شفافیت ارتباطات را حل کرده و نیاز به دانش قبلی از مکان سرورها
را از بین می‌برد، که به افزایش مقیاس‌پذیری سیستم و پنهان‌سازی جزئیات زیرین
پردازنده‌ها و ارتباطات کمک می‌کند. اگرچه \lr{Object Request Broker}های
تجاری به خوبی از این الگو پشتیبانی می‌کنند، اما ممکن است منابع بیشتری
نسبت به سیستم‌های کوچکتر نیاز داشته باشند، که در این
موارد می‌توان از \lr{ORB}های کوچکتر یا پیاده‌سازی سفارشی استفاده کرد.
\end{RTL}
\begin{figure}[h!]
\centering
\begin{tikzpicture}
    \lr{
        \umlclass[x=10]{LocalNotificationHandle}{}{
            }
            \umlclass[x=5, y=2]{Data}{
                }{
            \lr{update()}\\
            \lr{getDataID()}\\
            \lr{getDataName()}
            }
            \umlclass[x=6, y=-7]{AbstractProxy}{
            }{
                \lr{subscribe()}\\
                \lr{unsubscribe()}
            }
                \umlclass[x=7.5, y=-2]{NotficationHandle}{}{
        }    
        \umlclass[x=10, y=-4]{RemoteNotificationHandle}{}{
        }   
        \umlclass[x=1, y=-4]{AbstractClient}{}{
        }    
        \umlclass[x=-1, y=-7]{ConcreteClient}{}{
        }
        \umlclass[x=13.5, y=-7]{AbstractServer}{}{
        }   
        \umlclass[x=13.5, y=-9.5]{ConcreteServer}{}{
            } 
        \umlclass[x=5, y=-12]{ClientSideProxy}{}{ 
                }
            \umlclass[x=10, y=-12]{ServerSideProxy}{}{  
                        }
            \umlclass[x=7, y=-14]{Broker}{}{
        }
            \umlclass[x=7, y=-17]{RemoteNotificationHandle1}{}{
        }
    \umlunicompo[attr2=1|clientData, pos2=1.6, geometry=|-, anchor1=140, pos1=0.1]{AbstractClient}{Data}
    \umluniassoc[mult1=*, attr2=1|clientProxy, geometry=|-, anchor1=50, pos2=0.99]{AbstractClient}{ClientSideProxy}
    \umlassoc[mult1=*, mult2=1]{ClientSideProxy}{ServerSideProxy}
    \umlinherit[anchor1=150]{ServerSideProxy}{AbstractProxy}
    \umlinherit[]{ClientSideProxy}{AbstractProxy}
    \umlunicompo[mult2=*, anchor1=99, anchor2=-150]{AbstractProxy}{NotficationHandle}
    \umlinherit[]{RemoteNotificationHandle}{NotficationHandle}
    \umlunicompo[mult2=1, anchor2=-100, anchor1=125]{AbstractProxy}{Data}
    \umluniassoc[mult1=1, attr2=ServerProxy|1, anchor1=-150]{AbstractServer}{ServerSideProxy}
    \umlinherit[]{ConcreteClient}{AbstractClient}
    \umlunicompo[geometry=|-, attr2=ServerData|1, pos2=1.8]{AbstractServer}{Data}
    \umlinherit[]{LocalNotificationHandle}{NotficationHandle}
    \umluniassoc[geometry=|-]{ServerSideProxy}{Broker}
    \umluniassoc[geometry=|-]{ClientSideProxy}{Broker}
    \umlunicompo[mult2=*]{Broker}{RemoteNotificationHandle1}
    \umlinherit[]{ConcreteServer}{AbstractServer}
}
\end{tikzpicture}
\caption{دیاگرام کلاس \lr{Broker}}
\label{distrBrokerClassDiag}
\end{figure}
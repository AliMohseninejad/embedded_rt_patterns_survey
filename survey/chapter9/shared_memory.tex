\subsubsection{الگوی \lr{Shared Memory}}
\label{distrSharedMemSec}
\begin{RTL}
این الگو \cite{ref4}
به پردازنده‌های متعدد امکان می‌دهد داده‌ها را با استفاده از یک
ناحیه حافظه مشترک که معمولاً توسط چیپ‌های \lr{RAM}
چند پورت تسهیل می‌شود، به اشتراک بگذارند. این الگو ساده و مفید است
وقتی که نیاز به اشتراک‌گذاری داده‌ها بین پردازنده‌ها وجود دارد،
بدون نیاز به پاسخ‌های فوری به پیام‌ها یا رویدادها. این راه‌حل معمولاً ترکیبی
از سخت‌افزار و نرم‌افزار است، به طوری که سخت‌افزار با ارائه \lr{Semaphore}های
تک‌سیکل \lr{CPU} و دسترسی به حافظه از درگیری‌ها جلوگیری می‌کند،
در حالی که نرم‌افزار دسترسی مطمئن را تضمین می‌کند.
برای داده‌های فقط خواندنی، ممکن است چنین مکانیزم‌های همزمانی لازم نباشد.
\end{RTL}
\begin{RTL}
این الگو برای سیستم‌هایی که نیاز به اشتراک‌گذاری مقادیر زیادی از داده‌ها
به صورت پایدار بین پردازنده‌ها دارند، مانند پایگاه‌های داده مشترک یا کد اجرایی،
ایده‌آل است. این الگو معمولاً در کاربردهایی که سخت‌افزار و نرم‌افزار به طور مشترک طراحی
می‌شوند، استفاده می‌شود و نه راه‌حل‌های تجاری آماده.
سخت‌افزار ممکن است تعداد بلوک‌های قفل‌پذیر را محدود کند و نرم‌افزار باید
\lr{Semaphore}ها را برای کنترل دسترسی مدیریت کند
و شرایط رقابت را برای اطمینان از نوشتن موفق بررسی کند.
\end{RTL}
\begin{RTL}
حافظه مشترک برای ذخیره داده‌های بزرگ که به چندین پردازنده با سربار
کم قابل دسترسی هستند، موثر است، اگرچه ممکن است تحویل پیام‌ها
به موقع نباشد زیرا کلاینت‌ها حافظه مشترک را برای به‌روزرسانی‌ها بررسی می‌کنند.
\end{RTL}
\begin{figure}[h!]
\centering
\begin{tikzpicture}
    \lr{
        \begin{umlpackage}[x=-2, y=-11.5]{Processor}
            \umlclass[y=-1]{DataClient}{}{
                }   
                \umlclass[x=1, y=-4]{DataSource}{
                }{
                    } 
                \umlclass[x=1, y=-8]{Sender}{
                }{
                    } 
                    \umlclass[y=-12]{Reciver}{     
                }{
                    } 
        \end{umlpackage}
        \begin{umlpackage}[x=8, y=-12]{SharedMemory}
                \umlclass[x=1, y=-0.4]{GlobalData}{}{
                    \lr{readWrite()}
                        } 
                \umlclass[x=-2, y=-3]{HardwarSemaphor}{}{
                        }   
                \umlclass[x=2, y=-12]{MessageQueue}{
                    \lr{insert}\\
                    \lr{Remove}
                }{     
                            } 
                \umlclass[x=2, y=-15]{Message}{
                }{
                                } 
                \end{umlpackage}
                \umluniassoc[mult1=1, mult2=*, anchor1=130, anchor2=-133, pos2=0.9, pos1=0.1]{Reciver}{DataClient}
                \umluniassoc[mult1=*, mult2=1]{DataSource}{Sender}
                \umluniassoc[mult1=1, mult2=1, anchor1=-27]{Reciver}{MessageQueue}
                \umluniassoc[mult1=*, mult2=1, geometry=-|, pos2=1.9]{Reciver}{HardwarSemaphor}
                \umlunicompo[mult1=*, mult2=1]{MessageQueue}{Message}
                \umluniassoc[mult1=1, mult2=1, geometry=|-, anchor1=50, anchor2=-15, pos2=1.9, pos1=0.1]{MessageQueue}{HardwarSemaphor}
                \umluniassoc[mult1=*, mult2=*, geometry=-|, anchor1=-20, pos2=1.8, pos1=0.1]{Sender}{MessageQueue}
                \umluniassoc[mult1=*, mult2=1, geometry=-|, anchor2=-150, pos2=1.8]{Sender}{HardwarSemaphor}
                \umluniassoc[mult1=1, mult2=1, geometry=|-]{GlobalData}{HardwarSemaphor}
                \umluniassoc[mult1=*, mult2=1, pos1=0.1, pos2=0.9]{DataClient}{GlobalData}
}
\end{tikzpicture}
\caption{دیاگرام کلاس \lr{Shared Memory}}
\label{distrSharedMemClassDiag}
\end{figure}
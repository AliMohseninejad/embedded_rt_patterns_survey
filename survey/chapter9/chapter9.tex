\subsection{الگوهای معماری توزیع}
\begin{RTL}
توزیع یک جنبه مهم در معماری است که سیاست‌ها،
رویه‌ها و ساختار سیستم‌هایی را که ممکن است به طور همزمان در چند
فضای آدرس وجود داشته باشند، تعریف می‌کند. معماری‌های توزیع
به دو نوع اصلی نامتقارن و متقارن تقسیم می‌شوند.
در معماری نامتقارن، اتصال اشیا به فضاهای آدرس در زمان طراحی
مشخص است، در حالی که در معماری متقارن این اتصال تا زمان اجرا
مشخص نمی‌شود. این بخش از \cite{ref4} به بررسی معماری‌های توزیع در سطح برنامه
کاربردی می‌پردازد و تمرکز آن بر چگونگی یافتن و ارتباط اشیا با یکدیگر است.
\end{RTL}
\subsubsection{الگوی \lr{Shared Memory}}
\subsubsection{الگوی \lr{Remote Method Call}}
\subsubsection{الگوی \lr{Observer}}
\label{distrObserverSec}
\begin{RTL}
این الگو همان \nameref{HWObserverSec} است که در
\cite{ref1} گفته شده‌است؛ اما در این‌جا تعریف خود را به کاربردها
ارتباط با سنسور محدود نکرده‌است و در ساختار سیستم‌های توزیع‌شده معنا پیدا می‌کند.
در اینجا ساختار الگو کاملا مشابه \nameref{HWObserverSec} است؛
اما کلاس‌های کلاینت، هر کدام می‌توانند به صورت توزیع‌شده در آدرسی متفاوت
قرار داشته باشند.
\end{RTL}
\subsubsection{الگوی \lr{Data Bus}}
\label{distrDataBusSec}
\begin{RTL}
این الگو \cite{ref4}
\nameref{distrObserverSec} را با ارائه یک گذرگاه
مشترک گسترش می‌دهد که در آن چندین سرور اطلاعات خود را منتشر می‌کنند و چندین
کلاینت رویدادها و داده‌ها را دریافت می‌کنند.
این الگو برای سیستم‌هایی که سرورها و کاربران زیادی
باید داده‌ها را به اشتراک بگذارند مناسب است
و توسط گذرگاه‌های سخت‌افزاری مانند گذرگاه \lr{CAN} پشتیبانی می‌شود.
گذرگاه داده (\lr{Data Buss})
به عنوان یک مرکز مشترک برای به اشتراک‌گذاری داده‌ها در پردازنده‌ها
عمل می‌کند و به کلاینت‌ها اجازه می‌دهد که داده‌ها
را دریافت‌کنند یا برای دریافت آنها مشترک شوند.
این الگو به عنوان یک پروکسی با یک مخزن داده متمرکز عمل
می‌کند و می‌تواند اشیاء داده مختلف را مدیریت کند.
گذرگاه داده بسیار قابل گسترش است و انواع داده‌های جدید را
می‌توان بدون تغییر ساختار اصلی در زمان اجرا اضافه کرد.
با این حال، مکان گذرگاه داده باید از پیش تعیین شده باشد و مدیریت ترافیک
آن ممکن است ظرفیت گره را برای انجام کارهای دیگر محدود کند.
این الگو برای معماری‌های متقارن که سرورها در پردازنده‌های کمتر قابل
دسترس قرار دارند، مؤثر است.
\end{RTL}
\subsubsection{الگوی \lr{Proxy}}
\label{distrProxySec}
\begin{RTL}
الگوی پروکسی با استفاده از یک کلاس جایگزین، سرور واقعی را از کلاینت
انتزاع می‌کند و جداسازی و پنهان‌سازی ویژگی‌های خاص سرور
از کلاینت‌ها را امکان‌پذیر می‌سازد. این الگو در سیستم‌های نهفته که
سرورها ممکن است در فضای آدرس‌های مختلف باشند بسیار مفید است
و به کلاینت‌ها اجازه می‌دهد بدون اطلاع از مکان سرور با آن تعامل کنند.
این انتزاع طراحی مشتریان را ساده می‌کند و تغییرات سیستم را بدون تغییر
در تعاملات کلاینت-سرور تسهیل می‌کند. الگوی پروکسی به مدیریت شفافیت
ارتباطات کمک کرده و روش تماس با سرورهای راه دور را محصور می‌کند.
این الگو ترافیک ارتباطات را با کاهش تعداد پیام‌های ارسال شده
در شبکه و استفاده از سیاست اشتراک برای انتقال داده کاهش می‌دهد.
\end{RTL}
\subsubsection{الگوی \lr{Broker}}
\label{distrBrokerSec}
\begin{RTL}
الگوی بروکر یک نسخه متقارن از \nameref{distrProxySec} است که برای شرایطی
طراحی شده که مکان کلاینت و سرورها در زمان طراحی مشخص نیست. این الگو
یک بروکر را معرفی می‌کند، که یک مخزن ارجاع شیء است
و برای هر دو کلاینت و سرورها قابل مشاهده است و
به کلاینت‌ها در یافتن سرورها کمک می‌کند. این امکان استقرار
معماری‌های متقارن مانند تعادل بار پویا را فراهم می‌کند. الگوی بروکر
مسائل شفافیت ارتباطات را حل کرده و نیاز به دانش قبلی از مکان سرورها
را از بین می‌برد، که به افزایش مقیاس‌پذیری سیستم و پنهان‌سازی جزئیات زیرین
پردازنده‌ها و ارتباطات کمک می‌کند. اگرچه \lr{Object Request Broker}های
تجاری به خوبی از این الگو پشتیبانی می‌کنند، اما ممکن است منابع بیشتری
نسبت به سیستم‌های کوچکتر نیاز داشته باشند، که در این
موارد می‌توان از \lr{ORB}های کوچکتر یا پیاده‌سازی سفارشی استفاده کرد.
\end{RTL}
\subsubsection{الگوی \lr{Data Bus}}
\label{distrDataBusSec}
\begin{RTL}
این الگو \cite{ref4}
\nameref{distrObserverSec} را با ارائه یک گذرگاه
مشترک گسترش می‌دهد که در آن چندین سرور اطلاعات خود را منتشر می‌کنند و چندین
کلاینت رویدادها و داده‌ها را دریافت می‌کنند.
این الگو برای سیستم‌هایی که سرورها و کاربران زیادی
باید داده‌ها را به اشتراک بگذارند مناسب است
و توسط گذرگاه‌های سخت‌افزاری مانند گذرگاه \lr{CAN} پشتیبانی می‌شود.
گذرگاه داده (\lr{Data Buss})
به عنوان یک مرکز مشترک برای به اشتراک‌گذاری داده‌ها در پردازنده‌ها
عمل می‌کند و به کلاینت‌ها اجازه می‌دهد که داده‌ها
را دریافت‌کنند یا برای دریافت آنها مشترک شوند.
این الگو به عنوان یک پروکسی با یک مخزن داده متمرکز عمل
می‌کند و می‌تواند اشیاء داده مختلف را مدیریت کند.
گذرگاه داده بسیار قابل گسترش است و انواع داده‌های جدید را
می‌توان بدون تغییر ساختار اصلی در زمان اجرا اضافه کرد.
با این حال، مکان گذرگاه داده باید از پیش تعیین شده باشد و مدیریت ترافیک
آن ممکن است ظرفیت گره را برای انجام کارهای دیگر محدود کند.
این الگو برای معماری‌های متقارن که سرورها در پردازنده‌های کمتر قابل
دسترس قرار دارند، مؤثر است.
\end{RTL}
\begin{figure}[h!]
\centering
\begin{tikzpicture}
    \lr{
        \umlclass[x=2, y=-1]{UnitType}{}{
            }
            \umlclass[x=5, y=2]{DataBus}{
                }{
            \lr{update()}\\
            \lr{getDataID()}\\
            \lr{getDataName()}
            }
            \umlclass[x=5, y=-6]{AbstractData}{
                \lr{DataID}\\
                \lr{DataName}\\
                \lr{InfoType}\\
                \lr{DataUnits}
            }{     
            }
                \umlclass[x=7, y=-1]{IDType}{}{
        }    
        \umlclass[x=9, y=-3]{DataType}{}{
        }   
        \umlclass[y=-11]{AbstractClient}{}{
        }    
        \umlclass[y=-14]{ConcreteClient}{}{
        }
        \umlclass[x=10, y=-11]{AbstractSubject}{}{
        }   
        \umlclass[x=10, y=-14]{ConcreteSubject}{}{
            } 
        \umlclass[x=5, y=-10]{ConcreteData}{
            \lr{value}
        }{        
                }
    \umluniassoc[mult1=*, geometry=|-, anchor1=140, pos1=0.1]{AbstractClient}{DataBus}
    \umluniaggreg[mult1=1, mult2=1, geometry=|-, pos2=1.7]{AbstractClient}{AbstractData}
    \umlinherit[]{ConcreteData}{AbstractData}
    \umldep[]{AbstractData}{UnitType}
    \umldep[]{AbstractData}{IDType}
    \umldep[anchor1=28]{AbstractData}{DataType}
    \umlunicompo[mult2=*]{DataBus}{AbstractData}
    \umluniassoc[mult1=*, geometry=|-, anchor1=30]{AbstractSubject}{DataBus}
    \umlunicompo[mult1=1, mult2=1, geometry=|-, pos2=1.7]{AbstractSubject}{AbstractData}
    \umlinherit[]{ConcreteClient}{AbstractClient}
    \umlinherit[]{ConcreteSubject}{AbstractSubject}
}
\end{tikzpicture}
\caption{دیاگرام کلاس \lr{Data Bus (Pull Version)}}
\label{distrDataBusPullClassDiag}
\end{figure}
\begin{figure}[h!]
\centering
\begin{tikzpicture}
    \lr{
        \umlclass[x=2, y=-1]{UnitType}{}{
            }
            \umlclass[x=5, y=2]{DataBus}{
                }{
            \lr{update()}\\
            \lr{subscribe()}\\
            \lr{subscribe()}\\
            \lr{unSubscribe()}\\
            \lr{unSubscribe()}
            }
            \umlclass[x=5, y=-6]{AbstractData}{
            }{
                \lr{DataID()}\\
                \lr{DataName()}\\
                \lr{InfoType()}\\
                \lr{DataUnits()}
            }
                \umlclass[x=7, y=-1]{IDType}{}{
        }    
        \umlclass[x=9, y=-3]{DataType}{}{
        } 
        \umlclass[x=10.5, y=6]{NotficationHandle}{}{
            } 
            \umlclass[x=5, y=6]{NotficationHandler}{}{
                } 
        \umlclass[x=-1, y=-4]{Listener}{
            \lr{DataID}
        }{
            }   
        \umlclass[y=-11]{AbstractClient}{}{
        }    
        \umlclass[y=-14]{ConcreteClient}{}{
        }
        \umlclass[x=10, y=-11]{AbstractSubject}{}{
        }   
        \umlclass[x=10, y=-14]{ConcreteSubject}{}{
            } 
        \umlclass[x=5, y=-10]{ConcreteData}{
            \lr{value} 
        }{
        }
    \umluniassoc[mult1=*,mult2=1, geometry=|-, anchor1=140, pos1=0.1, pos2=1.9]{AbstractClient}{DataBus}
    \umluniaggreg[mult1=1, mult2=1, geometry=|-, pos2=1.7]{AbstractClient}{AbstractData}
    \umlinherit[]{ConcreteData}{AbstractData}
    \umldep[]{AbstractData}{UnitType}
    \umldep[]{AbstractData}{IDType}
    \umldep[anchor1=28]{AbstractData}{DataType}
    \umlunicompo[mult2=*]{DataBus}{AbstractData}
    \umluniassoc[mult1=*,mult2=1, geometry=|-, anchor1=30, pos2=1.9]{AbstractSubject}{DataBus}
    \umlunicompo[mult1=1, mult2=1, geometry=|-, pos2=1.7]{AbstractSubject}{AbstractData}
    \umlinherit[]{ConcreteClient}{AbstractClient}
    \umlinherit[]{ConcreteSubject}{AbstractSubject}
    \umlunicompo[mult2=*]{NotficationHandler}{NotficationHandle}
    \umlunicompo[mult2=*]{DataBus}{NotficationHandler}
    \umluniassoc[mult2=1, geometry=|-, anchor1=140, pos1=0.1]{AbstractClient}{Listener}    
}
\end{tikzpicture}
\caption{دیاگرام کلاس \lr{Data Bus (Push Version)}}
\label{distrDataBusPushClassDiag}
\end{figure}
\subsubsection{الگوی \lr{Proxy}}
\label{distrProxySec}
\begin{RTL}
الگوی پروکسی \cite{ref4}
با استفاده از یک کلاس جایگزین، سرور واقعی را از کلاینت
انتزاع می‌کند و جداسازی و پنهان‌سازی ویژگی‌های خاص سرور
از کلاینت‌ها را امکان‌پذیر می‌سازد. این الگو در سیستم‌های نهفته که
سرورها ممکن است در فضای آدرس‌های مختلف باشند بسیار مفید است
و به کلاینت‌ها اجازه می‌دهد بدون اطلاع از مکان سرور با آن تعامل کنند.
این انتزاع طراحی مشتریان را ساده می‌کند و تغییرات سیستم را بدون تغییر
در تعاملات کلاینت-سرور تسهیل می‌کند. الگوی پروکسی به مدیریت شفافیت
ارتباطات کمک کرده و روش تماس با سرورهای راه دور را محصور می‌کند.
این الگو ترافیک ارتباطات را با کاهش تعداد پیام‌های ارسال شده
در شبکه و استفاده از سیاست اشتراک برای انتقال داده کاهش می‌دهد.
\end{RTL}
\begin{figure}[h!]
\centering
\begin{tikzpicture}
    \lr{
        \umlclass[x=10]{LocalNotificationHandle}{}{
            }
            \umlclass[x=5, y=2]{Data}{
            \lr{value}
                }{
            }
            \umlclass[x=6, y=-7]{AbstractProxy}{
            }{
                \lr{subscribe()}\\
                \lr{unSubscribe()}
            }
                \umlclass[x=7.5, y=-2]{NotficationHandle}{}{
        }    
        \umlclass[x=10, y=-4]{RemoteNotificationHandle}{}{
        }   
        \umlclass[x=1, y=-3]{AbstractClient}{}{
            \lr{voidAccept()}\\  
        }    
        \umlclass[x=-1, y=-7]{ConcreteClient}{}{
        }
        \umlclass[x=13.5, y=-8]{AbstractServer}{}{
        }   
        \umlclass[x=14, y=-10.5]{ConcreteServer}{}{
            } 
        \umlclass[x=5, y=-13]{ClientSideProxy}{}{
                }
            \umlclass[x=10, y=-13]{ServerSideProxy}{}{
                    \lr{send()}  
                        }
    \umlunicompo[attr2=1|clientData, pos2=1.6, geometry=|-, anchor1=140, pos1=0.1]{AbstractClient}{Data}
    \umluniassoc[mult1=*, attr2=1|clientProxy, geometry=|-, anchor1=50, pos2=0.99]{AbstractClient}{ClientSideProxy}
    \umlassoc[mult1=*, mult2=1]{ClientSideProxy}{ServerSideProxy}
    \umlinherit[anchor1=150]{ServerSideProxy}{AbstractProxy}
    \umlinherit[]{ClientSideProxy}{AbstractProxy}
    \umlunicompo[mult2=*, anchor1=99, anchor2=-150]{AbstractProxy}{NotficationHandle}
    \umlinherit[]{RemoteNotificationHandle}{NotficationHandle}
    \umlunicompo[mult2=1, anchor2=-100, anchor1=125]{AbstractProxy}{Data}
    \umluniassoc[mult1=1, attr2=ServerProxy|1, anchor1=-150]{AbstractServer}{ServerSideProxy}
    \umlinherit[]{ConcreteClient}{AbstractClient}
    \umlunicompo[geometry=|-, attr2=ServerData|1, pos2=1.8]{AbstractServer}{Data}
    \umlinherit[]{LocalNotificationHandle}{NotficationHandle}
    \umlinherit[]{ConcreteServer}{AbstractServer}
}
\end{tikzpicture}
\caption{دیاگرام کلاس \lr{Proxy}}
\label{distrProxyClassDiag}
\end{figure}
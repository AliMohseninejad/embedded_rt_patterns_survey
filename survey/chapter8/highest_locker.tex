\subsubsection{الگوی \lr{Highest Locker}}
\label{resourceHighestLockerSec}
\begin{RTL}
این الگو \cite{ref4}
به هدف کاهش وارونگی اولویت، سقف اولویتی برای هر منبع تعیین می‌کند.
وظیفه‌ای که مالک منبع است، در بالاترین سقف اولویت از بین همه
منابعی که در اختیار دارد اجرا می‌شود، به شرطی که وظایف
با اولویت بالاتر را مسدود کند. این الگو که نوعی تصحیح شده از
\nameref{resourcePriorInheritSec} است، وارونگی اولویت را به یک سطح
واحد محدود می‌کند، به شرطی که وظایف در حین داشتن منابع خود را معلق نکنند.
بر خلاف \nameref{resourcePriorInheritSec}، این الگو مانع از مسدودسازی
زنجیره‌ای در صورت پیش‌دستی وظیفه‌ای در حین مالکیت منبع می‌شود.
\end{RTL}
\begin{RTL}
این الگو سقف اولویت‌ها را در زمان طراحی با شناسایی بالاترین اولویت
بین کلاینت‌های هر منبع و افزودن یک واحد به آن تعریف می‌کند.
این الگو به طور موثری وارونگی اولویت را محدود می‌کند، اما می‌تواند منجر
به بلوکه شدن بیشتر در این سطح واحد نسبت به روش‌های دیگر شود.
به عنوان مثال، اگر یک وظیفه با اولویت پایین یک منبع با سقف اولویت بالا
را قفل کند، وظایف با اولویت متوسط ممکن است بیشتر مسدود شوند.
برای مدیریت این وضعیت، می‌توان افزایش اولویت را تا زمانی که
وظیفه دیگری تلاش برای قفل کردن منبع کند، به تعویق انداخت.
\end{RTL}
\begin{RTL}
این الگو از \lr{Deadlock} جلوگیری می‌کند،
به شرطی که تسک‌ها در حین داشتن منابع خود را معلق نکنند،
زیرا اولویت تسک قفل‌کننده بالاتر از دیگر کلاینت‌های منبع است.
با این حال، سربار محاسباتی در مدیریت سقف اولویت‌ها
و اطمینان از اجرای صحیح تسک‌ها بدون تعلیق وجود دارد.
\end{RTL}
\subsubsection{الگوی \lr{Priority Inheritance}}
\label{resourcePriorInheritSec}
\begin{RTL}
الگوی \lr{Priority Inheritance} برای کاهش وارونگی اولویت
در سیستم‌های بی‌درنگ نهفته طراحی شده است که با تنظیم اولویت‌های
وظایفی که منابع را قفل می‌کنند، انجام می‌شود.
این الگو، اگرچه کامل نیست، اما وارونگی اولویت را با سربار اجرایی
نسبتاً کم به حداقل می‌رساند. وارونگی اولویت می‌تواند به خرابی‌های
سیستم منجر شود که تشخیص آنها دشوار است، زیرا وظایف گاهی
مهلت‌ها را از دست می‌دهند بدون اینکه علل واضحی داشته باشند.
این الگو تضمین می‌کند که یک وظیفه با اولویت بالا، در بدترین حالت،
تنها توسط یک وظیفه با اولویت پایین‌تر مسدود می‌شود و
به این ترتیب وارونگی اولویت بی‌نهایت را حل می‌کند.
\end{RTL}
\begin{RTL}
هنگامی که چندین منبع قفل شده باشند، مسدودسازی زنجیره‌ای
رخ می‌دهد، به طوری که یک وظیفه دیگری را در یک زنجیره
مسدود می‌کند. با این حال، این الگو به طور قابل توجهی مسدودسازی بی‌نهایت
را محدود می‌کند، به طوری که تعداد وظایف مسدود شده در
هر زمان کمتر از تعداد وظایف و منابع قفل شده است.
سربار زمانی از مدیریت اولویت‌های وظایف در هنگام مسدودسازی
و رفع مسدودسازی ناشی می‌شود، اما اگر مسدودسازی نادر باشد،
سربار کمی باقی می‌ماند.
\end{RTL}
\begin{RTL}
این الگو مشکلات \lr{Deadlock} را حل نمی‌کند و می‌تواند باعث ایجاد سربار ناشی
از جابه‌جایی وظایف و تنظیمات اولویت شود، به ویژه اگر رقابت بر سر منابع بالا باشد.
\end{RTL}
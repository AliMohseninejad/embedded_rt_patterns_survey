\subsubsection{الگوی \lr{Cyclic Executive}}
\label{scheduleCyclicExecSec}
\begin{RTL}
این الگو یکی از ساده‌ترین روش‌های زمانبندی در سیستم‌ها است. در این روش،
هر تسک شانس مساوی برای اجرا شدن دارد و تمامی تسک‌ها در یک حلقه
بی‌نهایت به صورت نوبتی جلو می‌روند.
این الگو در دو موقعیت مشخص کاربرد دارد. موقعیت اول زمانی است که سیستم
مورد بررسی یک سیستم نهفته بسیار کوچک است و می‌خواهیم بدون نیاز به
الگوریتم‌های پیچیده زمانبندی به یک ساختار شبه‌هم‌زمان برسیم.
موقعیت دوم زمانی است که سیستم مورد بررسی یک سیستم بسیار امن است و
می‌خواهیم به طور قطع از انجام درست فرایند برنامه‌ریزی برای تسک‌ها و تحقق
ددلاین‌ها مطمئن باشیم.
ساختار کلاسی این الگو در شکل \ref{scheduleCyclicExecSecClassDiag}
رسم شده‌است.
\end{RTL}
\begin{figure}[h!]
\centering
\begin{tikzpicture}
\lr{
    \umlclass{CyclicExecutive}{}{
        \lr{controlLoop()}
    }
    \umlclass[x=8]{AbstractCEThread}{}{
        \lr{run()}
    }
    \umlclass[x=8, y=-5]{ConcreteCEThread}{}{}
\umluniassoc[mult2=*]{CyclicExecutive}{AbstractCEThread}
\umlimpl[]{ConcreteCEThread}{AbstractCEThread}
}
\end{tikzpicture}
\caption{دیاگرام کلاس \lr{Cyclic Executive}}
\label{scheduleCyclicExecSecClassDiag}
\end{figure}
\begin{RTL}
کلاس \lr{CyclicExecutive} با داشتن یک حلقه تکرار
همیشگی، تابع \lr{run} را از هر یک از \lr{AbstractCEThread}هایی
که وجود دارد صدا می‌زند. \footnote{در \cite{ref1}، یک کلاس دیگر نیز
با نام \lr{CycleTimer} وجود دارد. اما به دلیل کاربر کم، در اینجا درباره
آن بحثی نمی‌کنیم.}
\end{RTL}
\subsection{الگوی \lr{Hardware Adapter}}
\label{HWAdapterSec}
\begin{RTL}
این الگو مشابه الگوی \lr{Adapter} که \lr{Gamma} و دیگران
\cite{ref3} معرفی کرده‌اند تعریف شده. استفاده از این الگو این اجازه را
می‌دهد که کلاینتی که انتظار یک رابط خاص با سخت‌افزار را دارد، بتواند با
سخت‌افزارهای مختلف بدون این‌که متوجه تفاوت‌های آن‌ها شود ارتباط بگیرد.
این الگو روی ساختار \nameref{HWProxySec} بنا شده‌است و
دیاگرام کلاس آن در شکل \ref{HWAdapterClassDiag} ترسیم شده‌است.
\end{RTL}
\begin{figure}[h!]
\centering
\begin{tikzpicture}
\lr{
    \umlclass{HardwareProxy}{
    device\_address
    }{
    initialize()\\
    configure()\\
    disable()\\
    access()\\
    mutate()
    }
    \umlclass[y=-6]{HardwareDevice}{
    \lr{}
    }{}
    \umlclass[x=6]{HardwareAdapter}{
    \lr{}
    }{
        clientService1() \\
        clientService2()
    }
    \umlinterface[x=6, y=4]{HardwareInterfaceToClient}{}{
        \umlvirt{clientService1()} \\
        \umlvirt{clientService2()}
    }
    \umlclass[x=13, y=4]{AdapterClient}{
        \lr{}
    }{}
\umlassoc[mult1=1, mult2=1]{HardwareProxy}{HardwareDevice}
\umluniassoc[mult1=1..*, mult2=1]{HardwareAdapter}{HardwareProxy}
\umlimpl[]{HardwareAdapter}{HardwareInterfaceToClient}
\umlassoc[mult1=1, mult2=1]{AdapterClient}{HardwareInterfaceToClient}
}
\end{tikzpicture}
\caption{دیاگرام کلاس \lr{Hardware Adapter}}
\label{HWAdapterClassDiag}
\end{figure}
\begin{RTL}
همانطور که در شکل \ref{HWAdapterClassDiag} دیده می‌شود،
کلاس کلاینت سرویس‌های مورد انتظار خود را از رابط
\lr{HardwareInterfaceToClient} انتظار دارد.
در این ساختار، کلاس آداپتور، سرویس‌های مورد انتظار کلاینت را به سرویس‌های ارائه‌شده
از طرف سخت‌افزار ترجمه می‌کند. این کار اجازه می‌دهد که در صورت تغییر سخت‌افزار
(و متناظرا پروکسی)، تنها با ایجاد پیاده‌سازی جدید برای رابط آداپتور، نیازی به تغییر
در کلاینت نباشد.
\end{RTL}
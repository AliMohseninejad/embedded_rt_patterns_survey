\subsection{الگوی \lr{Debouncing}}
\label{HWDebouncingSec}
\begin{RTL}
در سخت‌افزار بسیاری از ورودی‌ها به صورت دکمه‌ها و سوییچ‌هایی هستند که بر اثر ایجاد
اتصال دو فلز با یکدیگر، باعث فعال‌شدن یک پایه شده و آغازگر یک عملیات در نرم‌افزار
نهفته هستند. اتصال این دو فلز با یکدیگر دارای تعدادی حالت میانی است. به این صورت
که اتصال با کمی لرزش همراه بوده و اتصال برای چند میلی‌ثانیه چند بار قطع و وصل
می‌شود. این قطع و وصل شدن، باعث می‌شود که نتوانیم حالت فعلی سخت‌افزار را به درستی
در نرم‌افزار ضبط کنیم.
\end{RTL}
\begin{RTL}
این الگو به ما کمک می‌کند که با صبرکردن برای یک مدت کوتاه، مقدار ورودی را
زمانی که پایدار شده‌است بخوانیم. با این کار دغدغه معتبر بودن مقدار خوانده‌شده را
در کلاینت نخواهیم داشت. دیاگرام کلاسی این الگو در شکل \ref{HWDebouncingClassDiag}
نمایش داده شده‌است.
\end{RTL}
\begin{figure}[h!]
\centering
\begin{tikzpicture}
\lr{
    \umlclass{BouncingDevice}{
        \lr{deviceState}
    }{
        \lr{sendEvent()} \\
        \lr{getState()}
    }
    \umlclass[x=7]{Debouncer}{
        \lr{oldState}
    }{
        \lr{eventReceive()}
    }
    \umlclass[x=7, y=-4]{ApplicationClient}{}{
        \lr{deviceEventReceive(dState)}
    }
    \umlclass[x=7, y=4]{DebouncingTimer}{}{
        \lr{delay(delayTime)}
    }
\umlassoc[mult1=1, mult2=1]{Debouncer}{BouncingDevice}
\umluniassoc[attr2=1|itsApplicationClient]{Debouncer}{ApplicationClient}
\umluniassoc[attr2=itsDebouncingTimer|1]{Debouncer}{DebouncingTimer}
}
\end{tikzpicture}
\caption{دیاگرام کلاس \lr{Debouncing}}
\label{HWDebouncingClassDiag}
\end{figure}
\begin{RTL}
در این ساختار، \lr{BouncingDevice} همان سخت‌افزار مورد بررسی است.
تابع \lr{sendEvent} می‌تواند یک نوع اینتراپت سخت‌افزاری باشد که نرم‌افزار را
از تغییر در سخت‌افزار باخبر می‌سازد و \lr{getState} می‌تواند یک عملیات
خواندن از حافظه باشد.
کلاس \lr{Debouncer} وظیفه ارائه حالت پایدار سخت‌افزار به کلاینت را دارد.
این کار با استفاده از یک کلاس زمان‌سنج انجام می‌شود که با ایجاد یک تاخیر نرم‌افزاری
تا پایدارشدن شرایط خروجی سخت‌افزار، خواندن حالت سخت‌افزار را به تعویق می‌اندازد.
\end{RTL}
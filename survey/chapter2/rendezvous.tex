\subsubsection{الگوی \lr{Rendezvous}}
\label{scheduleRendezvousSec}
\begin{RTL}
این الگو \cite{ref1} زمانی کاربرد دارد که شروط لازم برای
سنکرون‌شدن تسک‌ها با یکدیگر پیچیده باشد.
در این صورت \nameref{scheduleGuardedCallSec} و
\nameref{scheduleQueuingSec} نمی‌توانند موثر واقع شوند و باید از
الگوی \lr{Rendezvous} استفاده کرد.
این الگو با استفاده از یک شیء مجزا برای تحقق بخشیدن به سنکرون‌سازی تسک‌ها،
تعدادی شرط تعریف می‌کند که با انجام آن‌ها، تسک‌ها سنکرون‌شده و آزاد می‌شوند.
این کار به این صورت انجام می‌شود که هر یک از تسک‌ها خود را پیش شیء
\lr{Rendezvous} رجیسترکرده و تا زمانی که این کلاس تصمیم بگیرد
متوقف می‌شوند.
\end{RTL}
\begin{figure}[h!]
\centering
\begin{tikzpicture}
    \lr{
        \umlclass[]{SynchronizingThread}{
            \lr{}
        }{
            \lr{}
            }
            \umlclass[x=6]{Rendezvous}{
            \lr{}
        }{
            \lr{synchronize()}\\
            \lr{reset()}
            }
            \umlclass[x=6, y=5]{Semaphore}{
                \lr{id}
            }{
                \lr{lock()}\\
                \lr{release()}
    }    
    \umluniassoc[mult1=2..*, mult2=1 , pos1=0.2]{SynchronizingThread}{Rendezvous}
    \umlcompo[attr2=itsBarrier|1, anchor1=119, anchor2=-120]{Rendezvous}{Semaphore}
    \umlcompo[attr2=1|itsMutex, pos2=0.8]{Rendezvous}{Semaphore}
    }
\end{tikzpicture}
\caption{دیاگرام کلاس \lr{Rendezvous}}
\label{scheduleRendezvousClassDiag}
\end{figure}
\begin{RTL}
همانطور که در شکل \ref{scheduleRendezvousClassDiag}
دیده می‌شود، کلاس \lr{SynchronizingThread} با فراخوانی
تابع \lr{synchronize} از \lr{Rendezvous} می‌خواهد خود را
سنکرون کند. این عملیات در \lr{Rendezvous} با بررسی چندین شرط
انجام می‌شود و در صورتی که شروط برقرار نباشند، این \lr{Thread} باید
بلوکه شود.
این الگو در عین سادگی، منعطف و قابل تطبیق است. این الگو زمانی کاربرد دارد
که می‌خواهیم تسک‌ها در زمان سنکرون‌شدن، متوقف شوند. اگر بخواهیم در توقف
اتفاق نیافتد، باید این الگو را با الگوهای دیگر ترکیب کنیم.
\end{RTL}
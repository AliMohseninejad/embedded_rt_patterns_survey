\subsubsection{الگوی \lr{Guarded Call}}
\label{scheduleGuardedCallSec}
\begin{RTL}
این الگو \cite{ref1} دسترسی به خدماتی را که ممکن است
در صورت استفاده همزمان توسط چندین \lr{Thread} تداخل کنند، مدیریت می‌کند و از مکانیزم قفل
برای جلوگیری از این تداخل استفاده می‌کند. \lr{Semaphore}ها انحصار متقابل
را در محیط چندوظیفه‌ای تضمین می‌کنند و دسترسی به‌موقع به خدمات را هنگامی
که توسط وظایف دیگر استفاده نمی‌شوند، فراهم می‌کنند. با این حال،
اگر با الگوهای دیگر ترکیب نشود، می‌تواند باعث وارونگی اولویت نامحدود شود.
این الگو نیاز به همگام‌سازی یا تبادل داده به‌موقع را برطرف می‌کند و
با جلوگیری از دسترسی همزمان، یکپارچگی داده‌ها را تضمین
می‌کند. استفاده اشتباه ممکن است منجر به وارونگی اولویت شود، اما
راه‌حل‌های پشتیبانی‌شده توسط \lr{RTOS} می‌توانند این مشکل را کاهش دهند.
ساختار کلاسی این الگو در شکل \ref{ConGuardCallClassDiag}
نمایش داده شده‌است.
\end{RTL}
\begin{figure}[h!]
\centering
\begin{tikzpicture}
    \lr{
        \umlclass[]{PreeptiveTask}{
            \lr{}
        }{
            \lr{run()}
            }
            \umlclass[x=5]{GuardedResource}{
            \lr{resourceValue}
        }{
            \lr{setResourceValue()}\\
            \lr{getResourceValue()}
            }
            \umlclass[x=5, y=-5]{Semaphore}{
                \lr{}
            }{
                \lr{lock()}\\
                \lr{release()}
    }    
    \umluniassoc[mult1=*, mult2=1 , pos1=0.2]{PreeptiveTask}{GuardedResource}
    \umlcompo[mult1=1, mult2=1]{GuardedResource}{Semaphore}
    }
\end{tikzpicture}
\caption{دیاگرام کلاس \lr{Guarded Call}}
\label{ConGuardCallClassDiag}
\end{figure}
\begin{RTL}
کلاس \lr{GuardedResource} با در اختیار داشتن \lr{Semaphore}ها
و فراخوانی \lr{lock} زمانی که یک تسک می‌خواهد از آن استفاده کند،
متوجه می‌شود که آیا باید به تسک اجازه استفاده از منابع را بدهد یا خیر.
زمانی تسک فعالی که در حال استفاده از \lr{GuardedResource} است،
کارش تمام می‌شود، \lr{GuardedResource} تابع \lr{release} را
برای \lr{Semaphore} مربوط به آن دسته تسک‌ها را صدا می‌زند و اکنون
منبع آزاد است تا زمانی که یک تسک جدید آن را قفل کند.
دقت شود که در ساختار این الگو کلاس‌هایی که توسط \lr{RTOS} ارائه می‌شوند
رسم نشده و صرفا هدف نشان‌دادن رفتار الگو است.
\end{RTL}
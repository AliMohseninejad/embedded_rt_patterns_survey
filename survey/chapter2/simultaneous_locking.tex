\subsubsection{الگوی \lr{Simultaneous Locking}}
\label{scheduleSimLockingSec}
\begin{RTL}
این الگو \cite{ref1} بر جلوگیری از بن‌بست تمرکز دارد
و اطمینان حاصل می‌کند که همه منابع مورد نیاز به‌طور همزمان قفل می‌شوند
یا هیچ‌کدام قفل نمی‌شوند. این روش با شکستن شرط نگه‌داشتن
برخی منابع در حین انتظار برای دیگران، از وقوع بن‌بست جلوگیری می‌کند،
اما ممکن است باعث افزایش تأخیر در اجرای وظایف شود.
این الگو به وظایف با اولویت بالاتر اجازه می‌دهد در صورت نیاز نداشتن
به منابع قفل‌شده اجرا شوند. با این حال، این الگو وارونگی اولویت
را حل نمی‌کند و ممکن است آن را بدتر کند مگر با استفاده
از الگوهای دیگر در
کنار این الگو مانند \nameref{resourcePriorInheritSec}.
\end{RTL}
\begin{figure}[h!]
\centering
\begin{tikzpicture}
    \lr{
        \umlclass[x=-3]{ResourceClient}{
            \lr{}
        }{
            \lr{}
            }
            \umlclass[x=5]{ResourceMaster}{
            \lr{}
        }{
            \lr{tryLock()}\\
            \lr{release()}
            }
            \umlclass[x=2, y=-5]{MultiMasteredResource}{
                \lr{}
            }{
                \lr{}\\
                \lr{accessResource()}
                }
                \umlclass[x=5, y=-10]{QueryMutex}{
                    \lr{mutexID}
                }{
                    \lr{lock()}\\
                    \lr{release()}\\
                    \lr{tryLock()}
    }    
    \umluniassoc[attr2=1|itsResourceMaster, pos2=0.6]{ResourceClient}{ResourceMaster}
    \umluniassoc[attr2=*|itsMultiMasteredResourceits, pos2=0.9, geometry=|-]{ResourceClient}{MultiMasteredResource}
    \umluniaggreg[attr2=*|itsMultiMasteredResource]{ResourceMaster}{MultiMasteredResource}
    \umlunicompo[attr2=1|itsPartMutex, geometry=|-, pos2=0.99]{MultiMasteredResource}{QueryMutex}
    \umlunicompo[attr2=1|itsQueryMutex, pos2=0.95 ]{ResourceMaster}{QueryMutex}
    }
\end{tikzpicture}
\caption{دیاگرام کلاس \lr{Simultaneous Locking}}
\label{scheduleSimLockingClassDiag}
\end{figure}
\begin{RTL}
همانطور که در شکل \ref{scheduleSimLockingClassDiag}
دیده می‌شود، هر \lr{ResourceClient} سرویس‌گیرنده از
\lr{MultiMasteredResource} است. در این ساختار، این کلاینت‌ها
حق استفاده از منابع را تا زمانی که نتوانسته‌اند کنترل \lr{ResourceMaster}
را بدست بگیرند، ندارند. زمانی که درخواست قفل‌کردن برای
\lr{ResourceMaster} می‌آید، این کلاس سعی می‌کند که تمامی منابع
را قفل کند، اگر در انجام این کار موفق بود، کلاینت اجازه دارد از منابع استفاده کند.
در غیر این صورت، تمامی منابعی که توسط \lr{ResourceMaster}
قفل شده‌بودند، دوباره آزاد می‌شوند و یک کد خطا به کلاینت برگردانده می‌شود.
\end{RTL}
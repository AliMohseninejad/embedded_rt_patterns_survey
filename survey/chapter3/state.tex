\subsubsection{الگوی \lr{State}}
\label{smStateSec}
\begin{RTL}
این الگو \cite{ref1} با واسپاری حالت سیستم به یک شیء
مجزا، وظیفه مدیریت حالت را به آن می‌دهد. در این الگو، تمامی رویدادهای دریافتی
به این شیء پاس داده می‌شوند و او با توجه به این که حالت بعدی را می‌شناسد،
خود را با شیء مربوط به حالت جدید جایگزین می‌کند.
در این ساختار، با اضافه‌شدن هر حالت جدید، باید یک کلاس جدید تعریف شود.
این الگو نسبت به \nameref{smStateTableSec} حافظه بیشتری اشغال
می‌کند اما با توزیع وظیفه مدیریت حالت بین کلاس‌های مختلف، پیاده‌سازی
ساده‌تری دارد. یکی دیگر از مزیت‌های استفاده از این الگو، این است که
این کلاس‌های حالت را می‌توان بین کلاینت‌های مختلف به اشتراک گذاشت.
\end{RTL}
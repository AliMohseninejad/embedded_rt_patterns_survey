\subsubsection{الگوی \lr{Polling}}
\label{HWPollingSec}
\begin{RTL}
این الگو یک روش دیگر برای دریافت داده‌ها از سنسورها است و زمانی استفاده می‌شود که
استفاده از \nameref{HWInterruptSec} ممکن نیست یا این که
داده‌هایی که می‌خواهیم ضبط کنیم آن قدر اضطراری نیستند و می‌توان برای دریافت آن‌ها
صبر کرد. عملکرد این الگو به این صورت است که با سرکشی‌کردن به صورت دوره‌ای داده‌ها
را دریافت می‌کنیم. حال این الگو در دو شکل بیان می‌شود: سرکشی داده‌ها به صورت
دوره‌ای و به صورت فرصتی.
در نوع اول با استفاده از یک تایمر، در زمان‌های مشخصی، برای دریافت داده‌های جدید
سرکشی می‌کنیم که ساختار کلاسی آن نیز
در شکل \ref{HWPeriodicPollingClassDiag} نشان داده شده‌است.
در نوع دوم زمانی عمل سرکشی را انجام می‌دهیم که برای سیستم ممکن باشد و
قیدهای زمانی سیستم به ما این اجاره را بدهد. ساختار کلاسی این نوع نیز در شکل
\ref{HWOpportunisticPollingClassDiag} رسم شده‌است.
\end{RTL}
\begin{figure}[h!]
\centering
\begin{tikzpicture}
\lr{
    \umlclass{Device}{
        \lr{data}\\
        \lr{deviceState}
    }{
        \lr{getData()} \\
        \lr{getState()}
    }
    \umlclass[x=8]{PeriodicPoller}{
        \lr{pollTime}
    }{
        \lr{startPolling()}\\
        \lr{stopPolling()}\\
        \lr{poll()}\\
        \lr{setPollTime()}
    }
    \umlclass[x=8, y=-5]{PollDataClient}{}{
        \lr{handleData()}\\
        \lr{handleDeviceState()}
    }
    \umlclass[x=8, y=6]{PollTimer}{
        \lr{oldVector}
    }{
        \lr{installInterruptHandler()}\\
        \lr{removeInterruptHandler()}\\
        \lr{startTimer()}\\
        \lr{stopTimer()}\\
        \lr{handleTimerInterrupt()}
    }
\umluniassoc[mult2=MAX\_POLL\_DEVICES, pos2=0.5]{PeriodicPoller}{Device}
\umluniassoc[mult2=MAX\_POLL\_DEVICES]{PeriodicPoller}{PollDataClient}
\umlassoc[mult1=1, mult2=1]{PeriodicPoller}{PollTimer}
}
\end{tikzpicture}
\caption{دیاگرام کلاس \lr{Periodic Polling}}
\label{HWPeriodicPollingClassDiag}
\end{figure}

\begin{figure}[h!]
\centering
\begin{tikzpicture}
\lr{
    \umlclass{Device}{
        \lr{data}\\
        \lr{deviceState}
    }{
        \lr{getData()} \\
        \lr{getState()}
    }
    \umlclass[x=8]{OpportunisticPoller}{}{
        \lr{poll()}
    }
    \umlclass[x=8, y=-5]{PollDataClient}{}{
        \lr{handleData()}\\
        \lr{handleDeviceState()}
    }
    \umlclass[x=8, y=6]{ApplicationProcessingElement}{}{
        \lr{applicationFunction()}
    }
\umluniassoc[mult2=MAX\_POLL\_DEVICES, pos2=0.5]{OpportunisticPoller}{Device}
\umluniassoc[mult2=MAX\_POLL\_DEVICES]{OpportunisticPoller}{PollDataClient}
\umluniassoc[mult2=1]{ApplicationProcessingElement}{OpportunisticPoller}
}
\end{tikzpicture}
\caption{دیاگرام کلاس \lr{Opportunistic Polling}}
\label{HWOpportunisticPollingClassDiag}
\end{figure}
\begin{RTL}
در این الگو کلاس \lr{Device} همان سخت‌افزار/حافظه/... هست که
می‌خواهیم داده‌هایش را دریافت کنیم. کلاس \lr{PollDataClient} نیز
کلاینتی است که می‌خواهد داده‌های \lr{Device} را دریافت کند.
در ساختار شکل \ref{HWPeriodicPollingClassDiag}، کلاس
\lr{PeriodicPoller} با سرکشی از \lr{Device} داده‌های
آن را دریافت می‌کند. این سرکشی زمانی انجام می‌شود که کلاس \lr{PollTimer}
تابع \lr{poll} را از \lr{PeriodicPoller} صدا بزند.
زمانی که فرمان \lr{startPolling} به بیاید، تایمر کار خود را شروع
می‌کند و در هر \lr{Interrupt}ی که تایمر می‌خورد، تابع \lr{poll} را
صدا می‌زند. با فراخوانی تابع \lr{stopPolling}، تایمر متوقف می‌شود.
در ساختار \ref{HWOpportunisticPollingClassDiag} تفاوت
در این است که تابع \lr{poll} زمانی صدا زده می‌شود که کلاس
\lr{ApplicationProcessingElement} بخواهد. یعنی زمانی که این
کلاس در فرایندهای خود لازم می‌بیند که لازم است داده‌های جدید از \lr{Device}
گرفته‌شود، این تابع صدا زده می‌شود.
\end{RTL}
\subsubsection{الگوی \lr{Observer}}
\label{HWObserverSec}
\begin{RTL}
یکی از پرکاربردترین الگوها در حوزه سیستم‌های نهفته، الگوی \lr{Observer} است.
این الگو به شیء‌های برنامه این اجازه را می‌دهد که به یک شی‌ء دیگر برای دریافت
اطلاعات گوش دهند. این به این معنی است که اگر یک کلاینت به دنبال دریافت
داده از یک سرور است، به جای این که هر دفعه درخواست دریافت داده‌ها را برای
سرور بفرستد، برای آن سرور درخواست عضویت فرستاده و سرور هرگاه که داده‌های جدید
در دسترس بودند، آن‌ها را برای کلاینت‌های عضوشده بفرستد. یکی از مهم‌ترین کاربردهای
این الگو در دریافت داده‌ها از سنسورها است.
یکی از قابلیت‌های خوب این الگو این است که کلاینت‌ها می‌توانند در زمان اجرای برنامه
عضویت خود را قطع یا ایجاد کنند. در شکل \ref{HWObserverClassDiag}
دیاگرام کلاس این الگو را می‌بینیم.
\end{RTL}
\begin{figure}[h!]
\centering
\begin{tikzpicture}
\lr{
    \umlinterface{AbstractClient}{}{
        \lr{accept(d: Datum)}
    }
    \umlclass[y=-3]{ConcreteClient}{}{}
    \umlclass[x=4, y=3]{Datum}{}{}
    \umlinterface[x=8]{AbstractSubject}{}{
        \lr{subscribe(a: acceptPtr)} \\
        \lr{unsubscribe(a: acceptPtr)} \\
        \lr{notify()}
    }
    \umlclass[x=8, y=-3]{ConcreteSubject}{}{}
    \umlclass[x=13, y=-3]{NotificationHandle}{
        \lr{aPtr: acceptPtr}
    }{}
\umluniassoc[geometry=|-, attr2=1|itsDatum, pos2=1.9, align2=left]{AbstractSubject}{Datum}
\umluniassoc[geometry=|-, attr2=1|itsDatum, pos2=1.9, align2=right]{AbstractClient}{Datum}
\umluniassoc[attr2=1|itsAbstractSubject, pos2=1, align2=right]{AbstractClient}{AbstractSubject}
\umlimpl{ConcreteSubject}{AbstractSubject}
\umlimpl{ConcreteClient}{AbstractClient}
\umlunicompo[geometry=-|, attr2=MAX\_SUBS|, pos2=1.9]{AbstractSubject}{NotificationHandle}
}
\end{tikzpicture}
\caption{دیاگرام کلاس \lr{Observer}}
\label{HWObserverClassDiag}
\end{figure}
\begin{RTL}
در این ساختار کلاینت‌ها با فرستادن یک اشاره‌گر به کلاس سابجکت، درخواست عضویت برای 
سرویس می‌فرستند. کلاس سابجکت نیز با ذخیره‌کردن اشاره‌گرهای مختلف از طرف کلاینت‌ها
زمانی که داده جدید آماده می‌شود، با فراخوانی تابع \lr{notify}، تابع \lr{accept}
که اشاره‌گرهایش را در \lr{NotificationHandle} قرار داده، با پاس دادن
ورودی به فرمت \lr{Datum} صدا می‌زند. اینگونه این داده برای تمامی کلاینت‌های
عضو سرویس فرستاده می‌شود.
کلاینت‌ها می‌توانند در حین اجرای برنامه، عضویت خود برای سرویس را لغو کنند.
دقت شود که خود کلاس‌‌های سابجکت
معمولا از نوع پروکسی هستند (\nameref{HWProxySec}).
\end{RTL}
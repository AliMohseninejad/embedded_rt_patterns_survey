\subsubsection{الگوی \lr{Interrupt}}
\label{HWInterruptSec}
\begin{RTL}
یکی از واحدهای مهم در سیستم‌های سخت‌افزاری، واحد \lr{Interrupt} است.
\lr{Interrupt} برای هندل‌کردن وقایعی است که توسط سخت‌افزار جرقه زده می‌شوند.
زمانی که یک \lr{interrupt} رخ می‌دهد، نرم‌افزار فرایند کاری اصلی خود را متوقف
کرده و یک فرایند رسیدگی به \lr{interrupt} اتفاق‌افتاده آغاز می‌شود. با انجام
این فرایند و رسیدگی به \lr{interrupt}، فرایند اصلی نرم‌افزار دوباره از سر گرفته
می‌شود. ساختار این الگو \cite{ref1} در شکل \ref{HWInterruptClassDiag}
نمایش داده شده‌است.
\end{RTL}
\begin{figure}[h!]
\centering
\begin{tikzpicture}
\lr{
    \umlclass{InterruptVectorTable <<File>>}{
        \lr{ISRAddress: vectorPtr[]}
    }{}
    \umlclass[x=7]{InterruptHandler}{
        \lr{oldVectors: vectorPtr}
    }{
        \lr{install()}\\
        \lr{deinstall()}\\
        \lr{handleInterrupt1()}\\
        \lr{handleInterrupt2()}\\
        \lr{handleInterrupt3()}\\
        \lr{handleInterrupt4()}
    }
    \umlclass[x=3, y=-5]{vectorPtr <<Type>>}{
        \lr{void (*vectorPtr)(void)}
    }{}
\umluniassoc[mult2=1]{InterruptHandler}{InterruptVectorTable <<File>>}
\umldep{InterruptHandler}{vectorPtr <<Type>>}
\umldep{InterruptVectorTable <<File>>}{vectorPtr <<Type>>}
}
\end{tikzpicture}
\caption{دیاگرام کلاس \lr{Interrupt}}
\label{HWInterruptClassDiag}
\end{figure}
\begin{RTL}
در این الگو، کلاس \lr{InterruptHandler} کار اصلی را انجام می‌دهد.
این کلاس دارای بردار \lr{Interrupt}هاست.
با فراخوانی تابع \lr{install} می‌توان این بردار را با یک بردار جدید جایگزین‌کرد.
این بردار در اصل تعدادی اشاره‌گر به توابعی است که در صورت بروز \lr{Interrupt}
باید فراخوانی شوند. با تابع \lr{deinstall} نیز می‌توان بردار را به حالت قبلی
برگرداند. فایل \lr{InterruptVectorTable} شامل یک لیست از
اشاره‌گرها به توابع \lr{Interrupt Service Routine} است.
و \lr{vectorPtr} صرفا یک نوع اشاره‌گر به تابع است که از نوع آن در
\lr{InterruptVectorTable} استفاده شده‌است.
\end{RTL}
\begin{RTL}
این الگو امکان پردازش سریع رویدادهای مهم را فراهم می‌کند.
با وقفه در پردازش معمولی (به شرطی که \lr{interrupt}ها غیرفعال نشده باشند)
باید در زمان پردازش حساس به زمان با احتیاط استفاده شود.
وقفه‌ها در حین اجرای \lr{Interrupt Service Routine (ISR)}
غیرفعال می‌شوند، بنابراین \lr{ISR}ها باید به سرعت اجرا شوند تا وقفه‌های دیگر
از دست نروند.
\lr{ISR}ها باید کوتاه باشند و در هنگام استفاده
از آن‌ها برای فراخوانی خدمات دیگر سیستم باید دقت
شود. برای به اشتراک گذاشتن داده‌ها، \lr{ISR}
ممکن است نیاز به صف‌بندی داده‌ها و بازگشت سریع داشته باشد تا برنامه در آینده
داده‌ها را در صف پیدا کند. این روش زمانی مفید است که جمع‌آوری
داده‌ها مهم‌تر از پردازش آن‌ها باشد. مشکلات زمانی رخ
می‌دهند که پردازش \lr{ISR} بیش از حد طولانی شود،
یا اشتباهاتی در پیاده‌سازی وقفه‌ها را غیرفعال کند، یا شرایط رقابتی یا بن‌بست
در منابع مشترک رخ دهد.
شرایط رقابتی زمانی رخ می‌دهد که نتیجه به ترتیب اجرای دستورات
بستگی دارد، اما این ترتیب ناشناخته است. بن‌بست زمانی است
که دو عنصر منتظر شرطی هستند که اصولاً اتفاق نمی‌افتد.
این مشکلات با استفاده از قفل‌های متقارن قابل حل
هستند، اما می‌توانند منجر به بن‌بست
شوند اگر \lr{ISR} منتظر قفل بماند. راه‌حل این است که
\lr{ISR} نباید منتظر قفل بماند و باید داده‌های جدید را رد کند
یا از منابع مشترک دوگانه استفاده کند.
\end{RTL}
\subsubsection{الگوی \lr{Interrupt}}
\label{HWInterruptSec}
\begin{RTL}
یکی از واحدهای مهم در سیستم‌های سخت‌افزاری، واحد \lr{Interrupt} است.
\lr{Interrupt} برای هندل‌کردن وقایعی است که توسط سخت‌افزار جرقه زده می‌شوند.
زمانی که یک \lr{interrupt} رخ می‌دهد، نرم‌افزار فرایند کاری اصلی خود را متوقف
کرده و یک فرایند رسیدگی به \lr{interrupt} اتفاق‌افتاده آغاز می‌شود. با انجام
این فرایند و رسیدگی به \lr{interrupt}، فرایند اصلی نرم‌افزار دوباره از سر گرفته
می‌شود. ساختار این الگو در شکل \ref{HWInterruptClassDiag}
نمایش داده شده‌است.
\end{RTL}
\begin{figure}[h!]
\centering
\begin{tikzpicture}
\lr{
    \umlclass{InterruptVectorTable <<File>>}{
        \lr{ISRAddress: vectorPtr[]}
    }{}
    \umlclass[x=7]{InterruptHandler}{
        \lr{oldVectors: vectorPtr}
    }{
        \lr{install()}\\
        \lr{deinstall()}\\
        \lr{handleInterrupt1()}\\
        \lr{handleInterrupt2()}\\
        \lr{handleInterrupt3()}\\
        \lr{handleInterrupt4()}
    }
    \umlclass[x=3, y=-5]{vectorPtr <<Type>>}{
        \lr{void (*vectorPtr)(void)}
    }{}
\umluniassoc[mult2=1]{InterruptHandler}{InterruptVectorTable <<File>>}
\umldep{InterruptHandler}{vectorPtr <<Type>>}
\umldep{InterruptVectorTable <<File>>}{vectorPtr <<Type>>}
}
\end{tikzpicture}
\caption{دیاگرام کلاس \lr{Interrupt}}
\label{HWInterruptClassDiag}
\end{figure}
\begin{RTL}
در این الگو، کلاس \lr{InterruptHandler} کار اصلی را انجام می‌دهد.
این کلاس دارای بردار \lr{Interrupt}هاست.
با فراخوانی تابع \lr{install} می‌توان این بردار را با یک بردار جدید جایگزین‌کرد.
این بردار در اصل تعدادی اشاره‌گر به توابعی است که در صورت بروز \lr{Interrupt}
باید فراخوانی شوند. با تابع \lr{deinstall} نیز می‌توان بردار را به حالت قبلی
برگرداند. فایل \lr{InterruptVectorTable} شامل یک لیست از
اشاره‌گرها به توابع \lr{Interrupt Service Routine} است.
و \lr{vectorPtr} صرفا یک نوع اشاره‌گر به تابع است که از نوع آن در
\lr{InterruptVectorTable} استفاده شده‌است.
\end{RTL}
\subsubsection{الگوی \lr{Hardware Proxy}}
\label{HWProxySec}
\begin{RTL}
این الگو \cite{ref1} با ایجاد یک رابط روی یک جزء سخت‌افزاری، یک دسترسی
مستقل از پیچیدگی‌های اتصال به سخت‌افزار برای کلاینت ایجاد می‌کند.
این الگو با معرفی یک کلاس به نام پروکسی بین سخت‌افزار و کلاینت،
باعث می‌شود که تمامی عملیات وابسته به سخت‌افزار در پروکسی انجام شود
و در صورت تغییر در سخت‌افزار، هیچ تغییری به کلاینت تحمیل نشود.
در این الگو بر روی یک جزء سخت‌افزاری، یک پروکسی قرار گرفته و
کلاینت‌های متعدد می‌توانند از آن سرویس بگیرند. لازم به ذکر است که ارتباط پروکسی
و سخت‌افزار بر پایه یک «رابط قابل آدرس‌دهی توسط نرم‌افزار» است. 
دیاگرام کلاس این الگو در شکل \ref{HWProxyClassDiag} رسم شده‌است.
\end{RTL}
\begin{figure}[h!]
\centering
\begin{tikzpicture}
\lr{
  \umlclass{HardwareProxy}{
    device\_address
  }{
    initialize()\\
    configure()\\
    disable()\\
    access()\\
    mutate()
  }
  \umlclass[y=-6]{HardwareDevice}{
    \lr{}
  }{}
  \umlclass[x=5]{ProxyClient}{
    \lr{}
  }{}
\umlassoc[mult1=1, mult2=1]{HardwareProxy}{HardwareDevice}
\umluniassoc[mult1=1..*, mult2=1]{ProxyClient}{HardwareProxy}
}
\end{tikzpicture}
\caption{دیاگرام کلاس \lr{Hardware Proxy}}
\label{HWProxyClassDiag}
\end{figure}
\begin{RTL}
همانطور که در شکل \ref{HWProxyClassDiag} دیده می‌شود، کلاس پروکسی توابع
مشخصی را در اختیار کلاینت‌ها قرار می‌دهد\footnote{توابع دیگری نیز در \cite{ref1}
گفته‌شده ولی اینجا تنها توابع \lr{public} کلاس پروکسی را بررسی می‌کنیم.}.
توضیحات مربوط به هر یک از توابع کلاس پروکسی در شکل زیر داده شده‌است:
\begin{itemize}
  \item \lr{initialize}:
  این تابع برای آماده‌سازی اولیه ارتباط با سخت‌افزار استفاده می‌شود و معمولا تنها یک بار
  صدا زده می‌شود.
  \item \lr{configure}:
  این تابع برای ارسال تنظیمات برای سخت‌افزار استفاده می‌شود. معمولا باید در سخت‌افزار
  تنظیماتی قرار داده‌شود که آن را قابل استفاده کند.
  \item \lr{disable}:
  این تابع برای غیرفعال‌کردن سخت‌افزار به صورت امن استفاده می‌شود.
  \item \lr{access}:
  این تابع برای دریافت اطلاعات از طرف سخت‌افزار استفاده می‌شود.
  \item \lr{mutate}:
  این تابع برای فرستادن اطلاعات به سمت سخت‌افزار استفاده می‌شود.
\end{itemize}
\end{RTL}
\begin{RTL}
این الگو بسیار رایج است و مزایای کپسوله‌سازی رابط سخت‌افزار و جزئیات
کدگذاری را فراهم می‌کند، به طوری که تغییرات رابط سخت‌افزار بدون نیاز به
تغییر در کلاینت‌ها انجام می‌شود. این کپسوله‌سازی می‌تواند تأثیر منفی
بر عملکرد زمان اجرا داشته باشد، زیرا کاربران از فرمت اصلی داده‌ها آگاه
نیستند. با این حال، آگاهی کلاینت‌ها از جزئیات کدگذاری باعث پیچیدگی
در نگه‌داشت سیستم می‌شود.
\end{RTL}